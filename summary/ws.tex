\documentclass[25pt]{sciposter}

\usepackage[T1]{fontenc}
\usepackage[utf8]{inputenc}

\usepackage{amsthm}

\usepackage[dvipsnames,usenames,svgnames,table]{xcolor} 
\usepackage{lipsum}
\usepackage{epsfig}
\usepackage{amsmath}
\usepackage{amssymb}
\usepackage[german]{babel}
\usepackage{geometry}
\usepackage{multicol}
\usepackage{graphicx}
\usepackage{tikz}
\usepackage{wrapfig}
\usepackage{gensymb}

\usepackage{empheq}

\usepackage{pgfplots}
\pgfplotsset{width=11cm,compat=1.9}


% for nice tableas
\usepackage{booktabs}

\graphicspath{ {img/} }

\geometry{
 landscape,
 a1paper,
 left=5mm,
 right=50mm,
 top=5mm,
 bottom=50mm,
 }
\usepackage{array}   % for \newcolumntype macro
\newcolumntype{L}{>{$}m{5.5cm}<{$}} % math-mode version of "l" column type

%BEGIN LISTINGDEF





\newcommand*\widefbox[1]{\fbox{\hspace{2em}#1\hspace{2em}}}
\newcommand{\limm}{\lim\limits_{n \to \infty}}
\newcommand{\limx}[1]{\lim\limits_{x \to #1}}
\newlength\dlf  % Define a new measure, dlf
\newcommand\alignedbox[2]{
% Argument #1 = before & if there were no box (lhs)
% Argument #2 = after & if there were no box (rhs)
&  % Alignment sign of the line
{
\settowidth\dlf{$\displaystyle #1$}
    % The width of \dlf is the width of the lhs, with a displaystyle font
\addtolength\dlf{\fboxsep+\fboxrule}
    % Add to it the distance to the box, and the width of the line of the box
\hspace{-\dlf}
    % Move everything dlf units to the left, so that & #1 #2 is aligned under #1 & #2
\boxed{#1 #2}
    % Put a box around lhs and rhs
}
}
\usepackage{graphicx,url}

%BEGIN TITLE
\title{\huge{Wahrscheinlichkeit und Statistik}}

\author{\large{David Zollikofer}}
%END TITLE

\usepackage{palatino}
%\usepackage{eulervm}
\usepackage{mathpazo}
% begin custom commands
\newcommand{\Q}{\mathbb{Q}}
\newcommand{\R}{\mathbb{R}}
\newcommand{\N}{\mathbb{N}}
\newcommand{\F}{\mathcal{F}}
\newcommand{\X}{\mathcal{X}}
\newcommand{\W}{\mathcal{W}}
\newcommand{\Var}{\operatorname{Var}}
\newcommand{\E}{\operatorname{E}}
%\newcommand{\exp}{\operatorname{exp}}


\newcommand{\norm}[1]{\left\lVert#1\right\rVert}
\usepackage[framemethod=TikZ]{mdframed}
\newenvironment{method}[1]{\begin{mdframed}[backgroundcolor=blue!10,innertopmargin=15pt, innerbottommargin=15pt,nobreak=true]
		\textbf{#1 }
	}
	{ 
	\end{mdframed}
}

\newenvironment{important}{\begin{mdframed}[backgroundcolor=red!50,innertopmargin=15pt, innerbottommargin=15pt, nobreak=true]
		\Large
	}
	{ 
	\end{mdframed}
}

\newenvironment{lemma}{\begin{mdframed}[backgroundcolor=gray!50,innertopmargin=15pt, innerbottommargin=15pt, nobreak=true]
		\Large
	}
	{ 
	\end{mdframed}
}

\newenvironment{thm}[1]{\begin{mdframed}[backgroundcolor=pink!50,innertopmargin=15pt, innerbottommargin=15pt, nobreak=true]
		\textbf{#1 }
	}
	{ 
	\end{mdframed}
}



\newenvironment{trick}[1]{\begin{mdframed}[backgroundcolor=PineGreen!50,innertopmargin=15pt, innerbottommargin=15pt, nobreak=true]
			\textbf{#1 }
	}
	{ 
	\end{mdframed}
}


\usepackage{todonotes}
\newcommand{\TODO}[1]{\todo[inline]{\Large TODO:  #1}}





\DeclarePairedDelimiter\abs{\left|}{\right|}%



\setlength\abovedisplayskip{0pt}

\renewcommand{\familydefault}{\rmdefault}

% end custom commands

\begin{document}





\maketitle


\begin{multicols}{3}



\section*{Wahrscheinlichkeitsrechnung}

\begin{method}{Wahrscheinlichkeitsraum}
	Wir definieren $(\Omega, \mathcal{F}, P)$ einen Wahrscheinlichkeitsraum, wobei $\Omega$ die Ereignismenge aus Elementarereignissen ist, $\F \subseteq 2^\Omega$ eine $\sigma$-Algebra und $P$ ein Wahrscheinlichkeitsmass.
\end{method}

\begin{method}{$\sigma$-Algebra}
	Wir nennen ein Mengensystem $\F\subseteq 2^\Omega$ eine $\sigma$-Algebra falls:
	\begin{itemize}
		\item $\Omega \in \F$
		\item $\forall A \in \F : A^c \in \F$
		\item für jede Folge $(A_n)_{n\in\N}$ mit $A_n \in \F$ so ist auch $\bigcup_{n=1}^\infty A_n \in \F $
	\end{itemize}

Sicherlich ist somit zum Beispiel die Potenzmenge $2^\Omega$ eine $\sigma$-Algebra.
\end{method}

\begin{method}{Wahrscheinlichkeitsmass} Wir definieren eine Abbildung $P: \F \to [0,1]$. Wir nennen $P[A]\in[0,1]$ die Wahrscheinlichkeit, dass $A$ eintritt. Die geforderten Kolmogorov Axiome sind:
	\begin{itemize}
		\item $P[A]\geq 0 \ \forall A \in \F$
		\item $P[\Omega] = 1$
		\item $P\left[ \bigcup_{i=1}^{\infty} A_i \right] = \sum_{i=1}^{\infty}P[A_i]$, sofern die $A_i \in \F$ paarweise disjunkt sind ($A_i \cap A_k = \varnothing$ wenn $i \neq k$) 
	\end{itemize}
\end{method}

\textbf{Beispiele (Formale Definitionen von Wahrscheinlichkeitsräumen)}



\subsection*{Grundlegende Prinzipien}



\begin{thm}{Additivität disjunkter Ereignisse}
	Seien $A_1, \ldots, A_n$ paarweise disjunkte Ereignisse, so gilt:
	
	\begin{align*}
	P[A_1\cup \cdots \cup A_n] &= P[A_1] + \cdots + P[A_n]
	\end{align*}
	
\end{thm}

\begin{thm}{Inklusion und Exklusionsprinzip}
	\begin{align*}
		P[A\cup B] =& P[A] + P[B] - P[A\cap B]\\
		P[A\cup B\cup C] =& P[A] + P[B] + P[C] - P[A\cap B] - P[B\cap C] \\
		&- P[B\cap C] + P[A\cap B \cap C]
	\end{align*}
\end{thm}



\subsection*{Bedingte Wahrscheinlichkeit}

\begin{method}{Bedingte Wahrscheinlichkeit}
Seien $A, B$ Ereignisse und $P[A] > 0$. Die bedingte Wahrscheinlichkeit von $B$ unter der Bedingung, dass $A$ eintritt, (gegeben $A$) ist definiert als:

\begin{align*}
	P[B|A] &= \frac{P[B \cap A]}{P[A]}
\end{align*}

\end{method}

\begin{thm}{Multiplikationsregel}
	Seien $A_1,\ldots A_n$ Ereignisse mit $P[A_i]>0$ (div by $0$ Problem), dann gilt:
	
	\begin{align*}
		&P[A_1 \cap A_2 \cap \cdots \cap A_n] = \\
		&P[A_1] \cdot P[A_2 | A_1] \cdot P[A_3|A_1 \cap A_2] \cdot \ldots \cdot P[A_n | A_1 \cap \cdots \cap A_{n-1}]
	\end{align*}
\end{thm}

\begin{thm}{Satz der totalen Wahrscheinlichkeit}
	 Seien $B_1, \ldots B_n$ eine Zerlegung von $\Omega$ (d.h. $\bigcup_{i=1}^n B_i = \Omega$ und $B_i \cap B_j = \varnothing$ für $i\neq j$), so gilt für ein beliebiges Ereignis $A$:
	 
	 \begin{align*}
	 	P[A] &= \sum_{i=1}^{n} P[A\cap B_i] = \sum_{i=1}^{n} P[A|B_i]\cdot P[B_i]
	 \end{align*}
	 
	 Insbesondere folgt daraus:
	 
	 \begin{align*}
	 	P[A] &= P[A\cap B] + P[A \cap B^c] = P[A|B]\cdot P[B] + P[A|B^c]\cdot P[B^c]
	 \end{align*}
\end{thm}


\begin{thm}{Bayes Theorem}
	Wenn $P[A],P[B].P[B^c]> 0$ so folgt:
	
	\begin{align*}
		P[B|A] = \frac{P[A\cap B]}{P[A]} &= \frac{P[A|B]\cdot P[B]}{P[A|B]\cdot P[B] + P[A|B^c]\cdot P[B^c]}
	\end{align*}
	
	respektive wenn $B_1, \ldots B_n$ eine Zerlegung von $\Omega$ mit $P[B_i]>0 \  \forall i$, dann gilt für ein Ereignis $A$ mit $P[A]>0$:
	
	\begin{align*}
		P[B_k|A] = \frac{P[A\cap B_k]}{P[A]} &= \frac{P[A|B_k]\cdot P[B_k]}{\sum_{i=1}^{n} P[A|B_i] \cdot P[B_i] }
	\end{align*}
\end{thm}


\subsection*{Unabhängigkeit}

\begin{method}{Unabhängigkeit}
	
	Wir nenne zwei Ereignisse $A$, $B$ unabhängig falls
	
	\begin{align*}
		P[A\cap B] &= P[A] \cdot P[B]
	\end{align*}
	
	
	Wir nennen $A_1,\cdots A_n$ unabhängig, falls für alle kombinationen von $A_i, \ldots A_j$ gilt dass
	
	\begin{align*}
		P\left[\bigcap_{i=1}^{m} A_{k_i} \right] = \prod_{i=1}^{m}P[A_{k_i}]
	\end{align*}
	
\end{method}





\section*{Kombinatorik}

\begin{itemize}
	\item \textbf{Auf wie viele Arten kann man $n$ Objekte (z.B. nebeneinander) anordnen?}
	
	Dies ist die Anzahl Permutationen von $n$ Elementen und ist $n!$.
	
	
	\item 	\textbf{Auf wie viele Arten kann man $k$ aus $n$ Objekten auswählen (mit $k\leq n$ ohne Zurücklegen)?}
	
	Dies ist die Anzahl Kombinationen ist $\binom{n}{k} = \frac{n!}{k!(n-k)!}$.
	
	
	\item \textbf{Wie viele Sequenzen der Länge $m$ kann ma mit den $n$ Symbolen bilden?}
	
	Dies ist die Anzahl der Variationen (mit Wiederholung) und ist $n^m$
\end{itemize}


\section*{Diskrete Zufallsvariablen und Verteilungen}



\begin{method}{Diskrete Zufallsvariable (ZV)}
	Sei $(\Omega, \F, P)$ ein diskreter Wahrscheinlichkeitsraum.
	
	Wir nennen 
	$$X:\Omega \to \W(X) = \{x_1,\ldots x_n\}\subseteq \R$$
	eine Zufallsvariable. mit $\W(X)$ der Wertebereich.
	 
	Zusätzlich definieren wir die \textit{Gewichtsfunktion} oder \textit{diskrete Dichte} von $X$ als 
	$$p_X(x_k) = P[X = x_k] = P[\{ \omega | X(\omega) = x_k \}]$$
	sowie auch die Verteilfunktion 
	$$F_x(t) = P[X\leq t] = P[\{\omega | X(\omega) \leq t\}]$$

\end{method}


\begin{method}{Erwartungswert}
	Sei $X$ eine diskrete Zufallsvariable mit gewichtsfunktion $p_X(x)$. Wir definieren den Erwartungswert von $X$ als:
	\begin{align*}
		\E[X] := \sum_{x_k\in \W(X)} x_k \cdot p_X(x)
	\end{align*}
\end{method}


\begin{thm}{Eigenschaften des Erwartungswerts}
\begin{itemize}
	\item Für $Y = g(X)$ für eine Funktion $g$, so gilt 
	\begin{align*}
		\E[Y] = \E[g(X)] = \sum_{x_k\in\W(X)} g(x_k)p_X(x_k)
	\end{align*}
	\item Falls $£X$ nur Werte in $\N_0$ annimmt gilt:
	\begin{align*}
		\E[X] = \sum_{j=1}^{\infty} P[X \geq j] = \sum_{l=0}^{\infty} P[X > l]
	\end{align*}
	\item Falls $X \leq Y$  (d.h. $X(w) \leq Y(w)$ für alle $w$) so gilt auch $\E[X] \leq \E[Y]$
	\item Für beliebige $a,b \in \R$ gilt $\E[aX + b] = a\E[X] + b$
\end{itemize}
\end{thm}



\begin{method}{Varianz \& Standardabweichung}
Sei $X$ eine diskrete Zufallsvariable. Ist $\E[X^2]$ konvergent, so gilt:
\begin{align*}
	\Var[X] := \E[(X-\E[X])^2]
\end{align*}
Zusätzlich definieren wir $\sigma(X) = \sqrt{\Var[X]}$ als die Standardabweichung.
\end{method}



\begin{thm}{Eigenschaften der Varianz}
	\begin{align*}
	\Var[X] &= \E[X^2] - (E[X])^2\\
	\Var[Y] &=\Var[aX+b] = a^2 \Var[X] \quad \text{mit } a,b\in\R , Y = aX+b
	\end{align*}
\end{thm}




\vfill


\newpage





%    _               _           _     
%   /_\  _ __   __ _| |_   _ ___(_)___ 
%  //_\\| '_ \ / _` | | | | / __| / __|
% /  _  \ | | | (_| | | |_| \__ \ \__ \
% \_/ \_/_| |_|\__,_|_|\__, |___/_|___/
%                       |___/           




% -------------------------- Ableitung --------------------------




\section{Ableitung}

\begin{method}{Ableitung}
	Sei $D \subset \R$ , $f:D \to  \R$ und $x_0$ ein Häufungspunkt von $D$. $f$ ist in $x_0$ differenzierbar, falls der Grenzwert 
	$$ \lim\limits_{x \to x_0} \frac{f(x) -f(x_0)}{x-x_0}$$
	existiert. Ist dies der Fall, wird der Grenzwert mit $f'(x_0)$ bezeichnet.\\
	Alternativ nutzt man auch $x = x_0 + h$
	\begin{align*}
			f'(x_0) = \lim\limits_{h \to 0} \frac{f(x_0 + h) - f(x_0)}{h}
	\end{align*}
\end{method}


\begin{method}{Weierstrass (Äquivalente Definitionen)}
Sei $f : D \to \R$, $x_0$ ein Häufungspunkt von $D$, dann sind folgende Aussagen äquivalent:
\begin{enumerate}
	\item $f$ ist in $x_0$ differenzierbar.
	\item Es gibt ein $c\in \R$ und $r : D \cup \{x_0\} \to \R$ mit:
	\begin{enumerate}
		\item $f(x) = f(x_0) + c(x-x_0) + r(x) (x-x_0)$
		\item $r(x_0) = 0$ mit $r$ stetig in $x_0$
	\end{enumerate}
Falls dies zutrifft ist $c=f'(x_0)$ eindeutig bestimmt.
\end{enumerate}
\end{method}

\textbf{(Beispiel) Per Definition Ableiten}

\begin{itemize}
	\item $f(x) = x^2$:
	\begin{align*}
		\frac{f(x) - f(x_0)}{x-x_0} &= 	\frac{x^2 - x_{0}^2}{x-x_0} = \frac{(x-x_0) (x+x_0)}{x-x_0} = x + x_0\\
		\lim_{x \to x_0} \frac{f(x)-f(x_0)}{x-x_0} &= \lim_{(x\to x_0)} x + x_0 = 2x_0
	\end{align*}
\end{itemize}



\begin{method}{Satz von Rolle}
	Sei $f: [a,b] \to \R$ stetig auf $(a,b)$ differenzierbar. Falls $f(a) = f(b)$, dann gibt es $\xi \in [a,b]$ mit $f'(\xi) = 0$
\end{method}
\textbf{Beweis (Satz von Rolle)} Aus dem Min-Max Satz folgt $\exists u,v \in [a,b]$ mit $f(u) \leq f(x) \leq f(v) \ \forall x \in [a,b]$. Falls einer der beiden in $(a,b)$ liegt nennen wir es $\xi$. Sonst gilt $f(a) = f(b)$ und dann $\xi = a$.


\begin{method}{Satz von Lagrange / Mittelwertsatz}
	Sei $f:[a,b] \to \R$ stetig mit $(a,b)$ differenzierbar. Dann gibt es $\xi \in (a,b)$ mit $$f(b) - f(a) = f'(\xi) (b-a)$$
	Dieser Satz ist auch bekannt als Mittelwertsatz. Die Aussage ist äquivalent zu: 
	\begin{align*}
		\exists x \in (a,b) : \quad f'(x) = \frac{f(b) -f(a) }{b-a}
	\end{align*}
\end{method}
\textbf{Beispiele (Lagrange)}
\begin{itemize}
	\item \textit{Zeige $|\sin(a) - \sin(b)| \leq |b-a|$:} Es folgt direkt dass $\exists c$: $\frac{\sin(b)- \sin(a)}{b-a} = \cos(c)$. Es folgt: $\cos(c) (b-a) = \sin(b) - \sin(a)$. Da aber $\cos(c) \leq 1$ folgt:
	$|b-a| \geq |\sin(b) - \sin(a)|$.
	\item \textit{Beweise: falls $f'(x) = 0 \ \forall x$, dann ist $f(x)$ auf $[a,b]$ konstant:} Aus Lagrange folgt dass für $x_1,x_2\in (a,b)$ beliebig: $0 = \frac{f(x_2)-f(x_1)}{x_2-x_1}$ dies impliziert $f(x_1) = f(x_2)$ $\forall x_1,x_2 \in (a,b)$.
\end{itemize}

\subsection*{Konvexität}

\begin{method}{Konvex}
	$f : I \to \R$ ist konvex (auf I) falls für alle $x \leq y$ $x,y \in I$ und $\lambda \in [0,1]$
	\begin{align*}
	f(\lambda x + (1-\lambda)y) \leq \lambda f(x) + (1-\lambda)f(y)
	\end{align*}
	Zudem gilt für $x_0 < x< x_1$ in $I$:
	\begin{align*}
	 \frac{f(x) - f(x_0)}{x-x_0} \leq \frac{f(x_1) - f(x)}{x_1 - x}
	\end{align*}
	Man beweist dies indem man $x = (1-\lambda) x_0 + \lambda x_1$ wählt und somit $\lambda = \frac{x-x_0}{x_1 - x_0}$
\end{method}




\subsection*{Wichtige Taylorapproximationen um $x=0$}
\begin{itemize}
	\item $\boxed{\frac{1}{1-x}}$ Für alle $x \in (1,0)$ gilt:
	\begin{align*}
	{\frac{1}{1-x}} &= 1 + x + x^2 + x^3 + x^4 + \cdots \\
	&= \sum_{n=0}^{\infty} x^n
	\end{align*}	
	
	\item $\boxed{e^x}$ Für alle $x \in \R$ gilt:
	\begin{align*}
		e^x &= 1 + x + \frac{x^2}{2!} + \frac{x^3}{3!} + \frac{x^4}{4!}\\
		&= \sum_{n=0}^{\infty} \frac{x^n}{n!}
	\end{align*}
	
	\item $\boxed{\cos(x)}$ Für alle $x\in R$ gilt:
	\begin{align*}
	\cos(x) &= 1 - \frac{x^2}{2!} + \frac{x^4}{4!} - \frac{x^6}{6!} + \frac{x^8}{8!} - \cdots  \\
	&= \sum_{n=0}^{\infty} (-1)^n \frac{x^{2n}}{(2n)!}
	\end{align*}
	
	\item $\boxed{\sin(x)}$ Für alle $x\in R$ gilt:
	\begin{align*}
	\sin(x) &=  x - \frac{x^3}{3!} + \frac{x^5}{5!} - \frac{x^7}{7!} + \frac{x^9}{9!} - \cdots\\
	&= \sum_{n=0}^{\infty} (-1)^n \frac{x^{2n+1}}{(2n+1)!} = \sum_{n=1}^{\infty} (-1)^{(n-1)} \frac{x^{2n-1}}{(2n-1)!}
	\end{align*}

	\item $\boxed{\ln(1+x)}$ Für alle $x\in (-1,1]$ gilt:
	\begin{align*}
	\ln(x+1) &= x - \frac{x^2}{2} + \frac{x^3}{3} - \frac{x^4}{4} + \frac{x^5}{5}- \cdots \\
	&= \sum_{n=1}^{\infty} (-1)^{(n+1)} \frac{x^n}{n}
	\end{align*}

	\item $\boxed{\arctan(x)}$ Für alle $x\in [-1,1]$ gilt:
	\begin{align*}
	\arctan(x) &= x - \frac{x^3}{3} + \frac{x^5}{5} - \frac{x^7}{7} + \frac{x^9}{9} - \cdots \\
	&= \sum_{n=0}^{\infty} (-1)^n \frac{x^{2n+1}}{2n+1}
	\end{align*}

	\item $\boxed{(1 + x)^\alpha}$ Für alle $x\in \R$ gilt:
	\begin{align*}
	(1 + x)^\alpha &=  1 + \alpha x + \frac{\alpha(\alpha-1)}{2!} x^2 + \cdots \\
	 &= \sum_{k=0}^{\infty} \; {\alpha \choose k} \; x^k 
	\end{align*}
	
	\item $\boxed{\sinh(x)}$ Für alle $x\in \R$ gilt:
	\begin{align*}
	\sinh(x) &= x + \frac{x^3}{6} + \frac{x^5}{120} + \mathcal{O}(x^7)\\
	&= \sum_{k=0}^{\infty}\frac{x^{1+2k}}{(1+2k)!}
	\end{align*}
	
	\item $\boxed{\cosh(x)}$ Für alle $x\in \R$ gilt:
	\begin{align*}
	\cosh(x) &= 1 + \frac{x^2}{2} + \frac{x^4}{24} + \frac{x^6}{720} +  \mathcal{O}(x^7)\\
	&= \sum_{k=0}^{\infty}\frac{x^{2k}}{(2k)!}
	\end{align*}
	

\end{itemize}






\subsection*{Fundamentalsatz}
\begin{method}{Stammfunktion}
Die Funktion $F(x) = \int_{a}^{x} f(t) dt$ ist in $[a,b]$ stetig und differenzierbar mit $F' = f$ wenn $a<b$ und $f:[a,b]\to R$ stetig ist. \\
\textit{Beweis:} Aus additivität folgt: $\int_{a}^{x_0} f(t) dt$ + $\int_{x_0}^{x} f(t) dt = \int_{a}^{x} f(t) dt$. Also $F(x) - F(x_0) = \int_{x_0}^{x} f(t) dt$. Per Mittelwertsatz sehen wir nun, dass es ein $\xi \in [x,x_0]$ gibt mit $\int_{x_0}^{x} f(t) dt = f(\xi) (x-x_0) $. Für $x \not = x_0$ folgt somit $\frac{F(x) - F(x_0)}{x-x_0} = f(\xi)$. Wegen der Stetigkeit von $f$ folgt: $\lim_{(x\to x_0)} \frac{F(x) - F(x_0)}{x-x_0} = f(x_0)$. \qed
\end{method}


\begin{method}{Fundamentalsatz der Differentialrechnung}
Sei $f:[a,b] \to \R$ stetig. Dann gibt es eine Stammfunktion $F$ von $f$, die bis auf eine additive Konstante eindeutig bestimmt ist und es gilt:
$$\int_{a}^{b} f(x) dx = F(b) - F(a)$$
\textit{Beweis:} Existenz folgt aus Stammfunktionssatz. Seien $F_1, F_2$ Stammfkt., dann gilt $F_1' - F_2' = 0$. Somit ist $F_1 - F_2 = C$ mit $F(x) = C + \int_{a}^{x} f(t) dt$. Es folgt auch $F(a) = \int_{a}^{a} f(t) dt + C$ und somit $F(a) = C$. Es folgt daraus $F(b)-F(a) = \int_{a}^{b} f(t) dt$ \qed 
\end{method}


\subsection*{Ableitung des Integrals}
Mit der Kettenregel folgt aus dem Fundamentalsatz:
\begin{align*}
\frac{d}{dx} \left( \int_{u(x)}^{v(x)} f(t)  dt \right) &= f(v)\frac{dv}{dx} - f(u)\frac{du}{dx}
\end{align*}

\section*{Integrale Ausrechnen}

\begin{important}
Integrationskonstante $C$ nicht vergessen!
\end{important}

\subsection*{Direkte Integrale}
Diese sind vom Typ $\int f(g(x)) g'(x) dx = F(g(x))$.

\subsection*{Partielle Integration}
\begin{method}{Partielle Integration}
\begin{align*}
	\int f' \cdot g \ dx = f \cdot g - \int f \cdot g' \  dx
\end{align*}
\end{method}



\subsection*{Integrale rationaler Funktionen}
\begin{method}{Partielle Integration}
	$$\int \frac{p(x)}{q(x)} dx$$
	Wenn nun $\deg(p) \geq \deg(q)$ dann machen wir eine Polynomdivision $p:q$, sonst mache eine Parzialbruchzerlegung
\end{method}

\subsection*{Substitutionsregel}
\begin{method}{Substitutionsregel}
Ist $f$ stetig und $g$ erfüllt:
\begin{align*}
	y = g(x) \iff x = g^{-1}(y)
\end{align*}
Dann gilt:
\begin{align*}
\int_a ^b f(g(x))g'(x) dx &= \int_{g(a)}^{g(b)} f(y) dy
\end{align*}
Als Merksatz gilt $dy = g'(x) dx$ respektive $dx = \frac{1}{t} dt$
\end{method}

\subsubsection*{Integrale der Form $\int F(e^x, \sinh(x), \cosh(x)) dx$}
Substituiere mit $e^x = t$, ($dx = \frac{1}{t} dt$)\\
\textbf{Beispiel:}
\begin{align*}
	\int \frac{e^{2x}}{e^x + 1} dx &= \int \frac{t^2}{t + 1 } \frac{1}{t} dt = \int\frac{t +1 - 1}{t+1} dt\\
	\int \frac{1}{\cosh(x)} dx &= \int \frac{1}{\frac{1}{2} (e^x + e^{-x})} dx = \int \frac{2}{t + \frac{1}{t}} \frac{1}{t} dt = \frac{2}{t^2 + 1} dt
\end{align*}

\subsubsection*{Integrale der Form $\int F(\log(x)) dx$}
Substituiere mit $\log(x) = t$, ($dx = e^t dt$)\\
\textbf{Beispiel:}
\begin{align*}
	\int (\log(x))^2 dx &= \int t^2 e^t dt = t^2 e^t - \int 2t e^t dt  \\
	&= x(\log(x))^2 -2x\log(x) + 2x + C
\end{align*}

\subsubsection*{Integrale der Form $\int F(\sqrt[\alpha]{Ax + B}) dx$}
Substituiere mit $t = \sqrt[\alpha]{Ax + B}$\\
\textbf{Beispiel:}
\begin{align*}
	\int \frac{1}{\sqrt{x} \sqrt{1-x}} &= \int \frac{1}{t \sqrt{1-t^2}} 2t dt = \int \frac{2}{\sqrt{1-t^2}}
\end{align*}

\subsubsection*{Integrale die $\sin, \cos, \tan$ in geraden Potenzen enthalten}
Substituiere mit $\tan(x) = t$, ($dx = \frac{1}{1+t^2} dt$). Es gilt zudem:

\begin{align*}
\sin^2(x) &= \frac{t^2}{1+t^2} & \cos^2(x) &= \frac{1}{1+t^2}
\end{align*}

\subsubsection*{Integrale die $\sin, \cos, \tan$ in ungeraden Potenzen enthalten}
Substituiere mit $\tan(\frac{x}{2}) = t$, ($dx = \frac{2}{1+t^2} dt$). Es gilt zudem:

\begin{align*}
\sin(x) &= \frac{2t}{1+t^2} & \cos(x) &= \frac{1-t^2}{1+t^2}
\end{align*}



\subsubsection*{Integrale mit $\sqrt{Ax^2 + Bx + C}$ im Nenner}
Mithilfe quadratischer Ergänzung auf einen der folgenden Fälle zurückführen:
\begin{align*}
\int \frac{1}{\sqrt{1-x^2}} dx &= \arcsin(x) + C\\
\int \frac{1}{\sqrt{x^2-1}} dx &= \operatorname{arcosh}(x) + C\\
\int \frac{1}{\sqrt{1+x^2}} dx &= \operatorname{arcsinh}(x) + C
\end{align*}




\subsubsection*{Integrale mit $\sqrt{Ax^2 + Bx + C}$ im Zähler}
Mithilfe quadratischer Ergänzung auf einen der folgenden Fälle zurückführen, dann substituieren
\begin{align*}
\int {\sqrt{1-x^2}} dx \quad &\text{subsitution: } x = \sin(t) \Leftarrow dx = \cos(t) dt\\
\int {\sqrt{x^2-1}} dx \quad &\text{subsitution: } x = \cosh(t) \Leftarrow dx = \sinh(t) dt \\
\int {\sqrt{1+x^2}} dx \quad &\text{subsitution: } x = \sinh(t) \Leftarrow dx = \cosh(t) dt
\end{align*}


% -------------------------- Partialbruchzerlegung ------------------------------




\vfill\null
\columnbreak

% to have more vertical space in the table.

{\renewcommand{\arraystretch}{1.5}
	\begin{table}[]
		\begin{tabular}{@{} p{.25\textwidth} p{.3\textwidth} p{.45\textwidth} @{}}
			\toprule
			Funktion & Ableitung & Bemerkung / Regel\\ \midrule
			$x$ & $1$ &   \\
			$x^2$& $2x$ &   \\
			$x^n$& $n\cdot x^{n-1}$ & $n \in \R$  \\
			$\frac{1}{x} = x^{-1}$ & $- \frac{1}{x^2}$ & \\
			$\sqrt{x} = x^{\frac{1}{2}}$ & $\frac{1}{2\sqrt{x}}$ & \\ 
			$\sqrt[n]{x} = x^{\frac{1}{n}}$ & $\frac{x^{\frac{1}{n} -1 }}{n}$ &  $\int x^{1/n} dx = \frac{n x^{1/n + 1}}{n+1} + C$\\ 
			$e^x$ & $e^x$ & \\
			$a^x$ & $\ln(a) \cdot a^x$& \\
			$x^x = e^{x\log(x)}$ & $x^x \cdot (\log(x) + 1)$ & Kettenregel $e^{x\log(x)}$\\
			$\ln(x)$ & $\frac{1}{x}$ & \\
			$x\ln(x) - x$ & $\ln(x)$ &  \\ \midrule
			$\sin(x)$ & $\cos(x)$ & \\
			$\cos(x)$ & $- \sin(x)$ & \\ 
			$\tan(x) = \frac{\sin(x)}{\cos(x)}$ & $\frac{1}{\cos^2(x)} = 1 + \tan^2(x)$ &\\
			$\cot(x) = \frac{\cos(x)}{\sin(x)}$ & - $\frac{1}{\sin^2(x)}$ & \\ 
			$\arcsin(x)$ & $\frac{1}{\sqrt{1 - x^2}}$ & $ \arcsin : [-1,1] \to [-\frac{\pi}{2},\frac{\pi}{2}]$\\
			$\arccos(x)$ & $ - \frac{1}{\sqrt{1-x^2}}$ & $\arccos : [-1,1] \to [0, \pi]$\\
			$\arctan(x)$ & $\frac{1}{1+x^2}$ & $\arctan:(-\infty, \infty) \to (- \frac{\pi}{2},\frac{\pi}{2})$\\
			$\operatorname{arccot}(x)$ & $ - \frac{1}{1+x^2} $ & $\operatorname{arccot} : (-\infty, \infty) \to (0,\pi)$\\
			\midrule
			$\cosh(x)$ & $\sinh(x)$ &\\
			$\sinh(x)$ & $\cosh(x)$ & \\
			$\tanh(x)$ & $\frac{1}{\cosh^2(x)}$ & \\
			$\operatorname{arsinh}(x)$ & $\frac{1}{\sqrt{1+x^2}}$ & $\forall x \in R$\\
			$\operatorname{arcosh}(x)$ & $\frac{1}{\sqrt{x^2 - 1}}$ & $\forall x \in (1, \infty)$\\		  $\operatorname{artanh}(x)$ & $\frac{1}{1-x^2}$ & $\forall x \in (-1,1)$\\
			\midrule
			$g(x) \cdot h(x)$ & $g(x) \cdot h'(x) + g'(x) \cdot h(x)$ & Produktregel\\
			$\left(g(x)\right)^n$ & $n \cdot \left( g(x) \right)^{n-1} \cdot g'(x)$ & Potenzregel\\
			$\frac{g(x)}{h(x)}$ & $\frac{ g'(x) \cdot h(x) - g(x)\cdot h'(x)}{\left(h(x)\right) ^2}$ & Quotientenregel\\
			$h(g(x))$ & $h'(g(x)) \cdot g'(x)$ & Kettenregel\\
			\bottomrule
		\end{tabular}
	\end{table}
}






% -------------------------- Sonstiges --------------------------





\section*{Sonstiges}
\begin{method}{Binomialsatz}
	$\forall x,y \in \mathbb{C}$, $n \geq 1$ gilt:
	$$(x+y)^n = \sum_{k=0}^{n} \binom{n}{k}x^k y^{n-k}$$
\end{method}

\begin{method}{ABC / Mitternachtsformel}
	\begin{align*}
	\text{Gegeben: } & ax^2 + bx + c = 0\\
	\text{Lösung: } & x_{1,2} = \frac{-b \pm \sqrt{b^2 -4ac}}{2a}
	\end{align*}
\end{method}


\begin{method}{Logarithmus Regeln}
\begin{align*}
	 \log _{b}(x\cdot y) &= \log _{b}(x)+\log _{b}(y)\\
	 \log_{b} (M^k) &= k \cdot \log_b (M)
\end{align*}
\end{method}


\begin{method}{Summenformeln}
	\begin{align*}
	\sum _{{k=1}}^{n}k &= {\frac  {n(n+1)}{2}}\\
	\sum_{k=1}^n (2k-1) &= n^2\\
	\sum _{{k=1}}^{n}k^{2} &= {\frac  {n(n+1)(2n+1)}{6}}
	\end{align*}
\end{method}


\begin{method}{Gerade \& Ungerade Funktion}
	Eine Funktion heisst:
	\begin{itemize}
		\item \textsc{Gerade} wenn $f(-x) = f(x)$
		\item \textsc{Ungerade} wenn $f(-x) = - f(x)$
	\end{itemize}
	Dabei sind $f(x) = 1$, $f(x) = |x|$, $f(x)=x^2$, $f(x) = \cos(x)$ alles gerade Funktionen.\\
	Im Gegenzug sind $f(x) = sgn(x)$, $f(x) = x$, $f(x) = \tan(x)$, $f(x) = \sin(x)$ ungerade Funktionen.
\end{method}


\begin{method}{Injektiv}
	\begin{align*}
	&\forall x_1,x_2 \in M : f(x_1) = f(x_2) \implies x_1 = x_2\\
	\text{or }  &x_1 \not = x_2 \implies f(x_1) \not = f(x_2)
	\end{align*}
	\textbf{Surjektiv}
	\begin{align*}
	\forall y \in N \exists x \in M : y = f(x)
	\end{align*}
\end{method}

\textbf{Umkehrsatz - Beispiel} Zeige dass $x + e^x$ bijektiv von $\R$ auf $\R$ abbildet. Es gilt $f'(x) = 1 + e^x > 0$, somit ist $f$ streng monoton wachsend in $\R$ und Umkehrbar. Weil $\lim_{x \to -\infty} f(x) = - \infty$ und $\lim_{x \to \infty} f(x) = \infty$ ist $f$ bijektiv von $\R$ nach $\R$



\begin{method}{Kreuzprodukt}
	\begin{align*}
	\vec{a}\times\vec{b}=	\begin{pmatrix}a_1 \\ a_2 \\ a_3\end{pmatrix}
	\times
	\begin{pmatrix}b_1 \\ b_2 \\ b_3 \end{pmatrix} &=	\begin{pmatrix}
	a_2b_3 - a_3b_2 \\
	a_3b_1 - a_1b_3 \\
	a_1b_2 - a_2b_1
	\end{pmatrix}
	\end{align*}
\end{method}



\subsection*{Wichtige Integrale}

\begin{itemize}	
	
	\item ${\displaystyle \int \sin ^{2}{ax}\,dx={\frac {x}{2}}-{\frac {1}{4a}}\sin 2ax+C={\frac {x}{2}}-{\frac {1}{2a}}\sin ax\cos ax+C}$
	
	\item ${\displaystyle \int \sin ^{n}{ax}\,dx=-{\frac {\sin ^{n-1}ax\cos ax}{na}}+{\frac {n-1}{n}}\int \sin ^{n-2}ax\,dx\qquad {\mbox{(for }}n>0{\mbox{)}}}$
	
	\item ${\displaystyle {\begin{aligned}\int x^{n}\sin ax\,dx&=-{\frac {x^{n}}{a}}\cos ax+{\frac {n}{a}}\int x^{n-1}\cos ax\,dx\end{aligned}}}$
	
	\item ${\displaystyle \int \cos ^{2}{ax}\,dx={\frac {x}{2}}+{\frac {1}{4a}}\sin 2ax+C={\frac {x}{2}}+{\frac {1}{2a}}\sin ax\cos ax+C}$
	
	\item ${\displaystyle \int \cos ^{n}ax\,dx={\frac {\cos ^{n-1}ax\sin ax}{na}}+{\frac {n-1}{n}}\int \cos ^{n-2}ax\,dx\qquad {\mbox{(for }}n>0{\mbox{)}}}$
	
	\item ${\displaystyle {\begin{aligned}\int x^{n}\cos ax\,dx&={\frac {x^{n}\sin ax}{a}}-{\frac {n}{a}}\int x^{n-1}\sin ax\,dx\end{aligned}}}$
	
	
	
	
	
	\item ${\displaystyle \int (\sin ax)(\cos ax)\,dx={\frac {1}{2a}}\sin ^{2}ax+C}$
	
	\item ${\displaystyle \int (\sin ^{n}ax)(\cos ax)\,dx={\frac {1}{a(n+1)}}\sin ^{n+1}ax+C\qquad {\mbox{(for }}n\neq -1{\mbox{)}}}$
	
	\item ${\displaystyle \int (\sin ax)(\cos ^{n}ax)\,dx=-{\frac {1}{a(n+1)}}\cos ^{n+1}ax+C\qquad {\mbox{(for }}n\neq -1{\mbox{)}}}$
	
	\item $ {\displaystyle {\begin{aligned}\int (\sin ^{n}ax)(\cos ^{m}ax)\,dx&=-{\frac {(\sin ^{n-1}ax)(\cos ^{m+1}ax)}{a(n+m)}}\\&+{\frac {n-1}{n+m}}\int (\sin ^{n-2}ax)(\cos ^{m}ax)\,dx\qquad {\mbox{(for }}m,n>0{\mbox{)}}\end{aligned}}} $
	
	\item $\int \sin^2(x) \cos^2(x) dx = \frac{1}{4}\int sin^2(2x) dx = \frac{1}{4} \int \frac{1-\cos(4x)}{2}dx = \frac{x}{8} - \frac{1}{8} \frac{\sin(4x)}{4} + C$  
	
\end{itemize}



\subsubsection*{Typische Integrale}

\begin{itemize}
	\item $\int \frac{1}{x} \,dx = \ln |x|$
	\item $\int \frac{1}{x^2} \,dx = -\frac{1}{x}$
	\item $\int \frac{1}{x+a} \,dx = \ln |x+a|$
	\item $\int \ln(x) \,dx = x(\ln(x) - 1)$
	\item $\int \ln(ax + b) \,dx = \frac{(a x+b) \ln (a x+b)-a x}{a}$
	\item $\int \frac{1}{(x+a)^2} \,dx = - \frac{1}{x+a}$
	\item $\int \frac{1}{\sqrt{x}} \,dx = 2 \sqrt{x}$
	\item $\int \sqrt{1-x^2} dx = \frac{\arcsin(x) + x \sqrt{1-x^2}}{2} + C$
	\item $\int \frac{1}{ax+b} \,dx = \frac{1}{a} \ln |ax+b|$
	\item $\int \frac{1}{1 + x^2} \,dx = \frac{1}{2} \ln |1 + x^2|$
	\item $\int(ax + b)^n \,dx = \frac{(ax + b)^{n+1}}{(n + 1)a}, (n \neq -1)$
	\item $\int x(ax+b)^n \,dx = \frac{(ax + b)^{n+2}}{(n+2)a^2} -
	\frac{b(ax+b)^{n+1}}{(n+1)a^2}$
	\item $\int \frac{ax + b}{px + q} \,dx = \frac{ax}{p} + \frac{bp - aq}{p^2} \ln
	|pq+q|$
	\item $\int \frac{1}{a^2 + x^2} \,dx = \frac{1}{a} \arctan(\frac{x}{a})$
	\item $\int \frac{1}{a^2 - x^2} \,dx = \frac{1}{2a} \ln \left | \frac{a+x}{a-x}
	\right |$
	\item $\int \sqrt{x} \,dx = \frac{2}{3}\sqrt{x^3}$
	%mühsamer kerl der teilweise in prüfungen verwendet wird. Kann man über subsitution von x mit sin(u) lösen.
	\item $\int \sqrt{1-x^2} \,dx = \frac{1}{2}\left( x\sqrt{1-x^2}+\frac{1}{\sin(x)} \right)$
	\item $\int a^{xb + c} \,dx = \frac{a^{bx + c}}{b \log(a)}$
\end{itemize}

\subsubsection*{Trionometrische Funktionen}
\begin{itemize}
	\item $\int \sin(ax) \,dx = -\frac{1}{a}\cos(ax)$
	\item $\int \cos(ax) \,dx = \frac{1}{a}\sin(ax)$
	\item $\int \sin(ax)^2 \,dx = \frac{x}{2} - \frac{sin(2ax)}{4a}$
	\item $\int \frac{1}{\sin^2 x} \,dx = -\cot x$
	\item $\int x \sin(ax) \,dx = \frac{\sin(ax)}{a^2} - \frac{x \cos(ax)}{a}$
	\item $\int \cos^2(ax) \,dx = \frac{x}{2} + \frac{\sin(2ax)}{4a}$
	\item $\int \frac{1}{\cos^2(x)} \,dx = \tan x$
	\item $\int \cos(ax) \,dx = \frac{\cos(ax)}{a^2} + \frac{x \sin(ax)}{a}$
	\item $\int \sin(ax) \cos(ax) \,dx = -\frac{\cos^2(ax)}{2a}$
	\item $\int \tan(ax) \,dx = - \frac{1}{a} \ln | \cos(ax) |$
\end{itemize}

\subsubsection*{Exponentialfunktion}
\begin{itemize}
	\item $\int e^{ax} \,dx = \frac{1}{a} e^{ax}$ 
	\item $\int x e^{ax} \,dx = e^{ax} \cdot \left ( \frac{ax - 1}{a^2} \right )$
	\item $\int x \ln(x) \,dx = \frac{1}{2} x^2 (\ln(x) - \frac{1}{2})$
	\item $\int_{-\infty}^\infty e^{-\frac{1}{a}x^2} \,dx = \sqrt{a \pi}$
\end{itemize}




\subsection*{Vektoranalysis}

\begin{align*}
	{\displaystyle \Delta f=\operatorname {div} \left(\operatorname {grad} \,f\right),}\\
	{\displaystyle \Delta f=\nabla \cdot (\nabla f)=(\nabla \cdot \nabla )f=\nabla ^{2}f.}\\
	\nabla f = \begin{pmatrix}
	\frac{\partial }{\partial x}\\
	\frac{\partial }{\partial y}\\
	\frac{\partial }{\partial z}
	\end{pmatrix} \cdot f
\end{align*}

\newpage

\end{multicols}
\end{document}
