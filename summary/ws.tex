\documentclass[25pt]{sciposter}

\usepackage[T1]{fontenc}
\usepackage[utf8]{inputenc}

\usepackage{amsthm}

\usepackage[dvipsnames,usenames,svgnames,table]{xcolor} 
\usepackage{lipsum}
\usepackage{epsfig}
\usepackage{amsmath}
\usepackage{amssymb}
\usepackage[german]{babel}
\usepackage{geometry}
\usepackage{multicol}
\usepackage{graphicx}
\usepackage{tikz}
\usepackage{wrapfig}
\usepackage{gensymb}
\usepackage{tocloft}
\usepackage{empheq}

\usepackage{pgfplots}
\pgfplotsset{width=11cm,compat=1.9}


% for nice tableas
\usepackage{booktabs}

\graphicspath{ {img/} }

\geometry{
 landscape,
 a1paper,
 left=5mm,
 right=50mm,
 top=5mm,
 bottom=50mm,
 }
\usepackage{array}   % for \newcolumntype macro
\newcolumntype{L}{>{$}m{5.5cm}<{$}} % math-mode version of "l" column type

%BEGIN LISTINGDEF





\newcommand*\widefbox[1]{\fbox{\hspace{2em}#1\hspace{2em}}}
\newcommand{\limm}{\lim\limits_{n \to \infty}}
\newcommand{\limx}[1]{\lim\limits_{x \to #1}}
\newlength\dlf  % Define a new measure, dlf
\newcommand\alignedbox[2]{
% Argument #1 = before & if there were no box (lhs)
% Argument #2 = after & if there were no box (rhs)
&  % Alignment sign of the line
{
\settowidth\dlf{$\displaystyle #1$}
    % The width of \dlf is the width of the lhs, with a displaystyle font
\addtolength\dlf{\fboxsep+\fboxrule}
    % Add to it the distance to the box, and the width of the line of the box
\hspace{-\dlf}
    % Move everything dlf units to the left, so that & #1 #2 is aligned under #1 & #2
\boxed{#1 #2}
    % Put a box around lhs and rhs
}
}
\usepackage{graphicx,url}

%BEGIN TITLE
\title{\huge{Wahrscheinlichkeit und Statistik}}

\author{\large{David Zollikofer}}
%END TITLE

\usepackage{palatino}
%\usepackage{eulervm}
\usepackage{mathpazo}
% begin custom commands
\newcommand{\Q}{\mathbb{Q}}
\newcommand{\R}{\mathbb{R}}
\newcommand{\N}{\mathbb{N}}
\newcommand{\F}{\mathcal{F}}
\newcommand{\X}{\mathcal{X}}
\newcommand{\W}{\mathcal{W}}
\newcommand{\Nor}{\mathcal{N}}
\newcommand{\U}{\mathcal{U}}
\newcommand{\Var}{\operatorname{Var}}
\newcommand{\E}{\operatorname{E}}
%\newcommand{\exp}{\operatorname{exp}}

\newcommand{\mc}{\mathcal}

% some shortcuts
\newcommand{\ds}{\displaystyle}
\newcommand{\arr}{\rightarrow}
\newcommand{\nop}[1]{}
\renewcommand{\hat}{\widehat}

% stuff for integrals
\newcommand{\intl}{\int\limits}
\newcommand{\rmd}{\mathrm{d}}


\newcommand{\norm}[1]{\left\lVert#1\right\rVert}
\usepackage[framemethod=TikZ]{mdframed}
\newenvironment{method}[1]{\begin{mdframed}[backgroundcolor=blue!10,innertopmargin=15pt, innerbottommargin=15pt,nobreak=true]
		\textbf{#1 }
	}
	{ 
	\end{mdframed}
}

\newenvironment{important}{\begin{mdframed}[backgroundcolor=red!50,innertopmargin=15pt, innerbottommargin=15pt, nobreak=true]
		\Large
	}
	{ 
	\end{mdframed}
}

\newenvironment{lemma}{\begin{mdframed}[backgroundcolor=gray!50,innertopmargin=15pt, innerbottommargin=15pt, nobreak=true]
		\Large
	}
	{ 
	\end{mdframed}
}

\newenvironment{thm}[1]{\begin{mdframed}[backgroundcolor=pink!20,innertopmargin=15pt, innerbottommargin=15pt, nobreak=true]
		\textbf{#1 }
	}
	{ 
	\end{mdframed}
}



\newenvironment{trick}[1]{\begin{mdframed}[backgroundcolor=PineGreen!50,innertopmargin=15pt, innerbottommargin=15pt, nobreak=true]
			\textbf{#1 }
	}
	{ 
	\end{mdframed}
}


\usepackage{todonotes}
\newcommand{\TODO}[1]{\todo[inline]{\Large TODO:  #1}}





\DeclarePairedDelimiter\abs{\left|}{\right|}%



\setlength\abovedisplayskip{0pt}

\renewcommand{\familydefault}{\rmdefault}

% end custom commands

\begin{document}





\maketitle



\begin{multicols}{3}

\part{Wahrscheinlichkeit}

\section{Wahrscheinlichkeitsrechnung}

\begin{method}{Wahrscheinlichkeitsraum}
	Wir definieren $(\Omega, \mathcal{F}, P)$ einen Wahrscheinlichkeitsraum, wobei $\Omega$ die Ereignismenge aus Elementarereignissen ist, $\F \subseteq 2^\Omega$ eine $\sigma$-Algebra und $P$ ein Wahrscheinlichkeitsmass.
\end{method}

\begin{method}{$\sigma$-Algebra}
	Wir nennen ein Mengensystem $\F\subseteq 2^\Omega$ eine $\sigma$-Algebra falls:
	\begin{itemize}
		\item $\Omega \in \F$
		\item $\forall A \in \F : A^c \in \F$
		\item für jede Folge $(A_n)_{n\in\N}$ mit $A_n \in \F$ so ist auch $\bigcup_{n=1}^\infty A_n \in \F $
	\end{itemize}

Sicherlich ist somit zum Beispiel die Potenzmenge $2^\Omega$ eine $\sigma$-Algebra.
\end{method}

\begin{method}{Wahrscheinlichkeitsmass} Wir definieren eine Abbildung $P: \F \to [0,1]$. Wir nennen $P[A]\in[0,1]$ die Wahrscheinlichkeit, dass $A$ eintritt. Die geforderten Kolmogorov Axiome sind:
	\begin{itemize}
		\item $P[A]\geq 0 \ \forall A \in \F$
		\item $P[\Omega] = 1$
		\item $P\left[ \bigcup_{i=1}^{\infty} A_i \right] = \sum_{i=1}^{\infty}P[A_i]$, sofern die $A_i \in \F$ paarweise disjunkt sind ($A_i \cap A_k = \varnothing$ wenn $i \neq k$) 
	\end{itemize}
\end{method}

\textbf{Beispiele (Formale Definitionen von Wahrscheinlichkeitsräumen)}



\subsection*{Grundlegende Prinzipien}



\begin{thm}{Additivität disjunkter Ereignisse}
	Seien $A_1, \ldots, A_n$ paarweise disjunkte Ereignisse, so gilt:
	
	\begin{align*}
	P[A_1\cup \cdots \cup A_n] &= P[A_1] + \cdots + P[A_n]
	\end{align*}
	
\end{thm}

\begin{thm}{Inklusion und Exklusionsprinzip}
	\begin{align*}
		P[A\cup B] =& P[A] + P[B] - P[A\cap B]\\
		P[A\cup B\cup C] =& P[A] + P[B] + P[C] - P[A\cap B] - P[B\cap C] \\
		&- P[B\cap C] + P[A\cap B \cap C]
	\end{align*}
\end{thm}



\subsection*{Bedingte Wahrscheinlichkeit}

\begin{method}{Bedingte Wahrscheinlichkeit}
Seien $A, B$ Ereignisse und $P[A] > 0$. Die bedingte Wahrscheinlichkeit von $B$ unter der Bedingung, dass $A$ eintritt, (gegeben $A$) ist definiert als:

\begin{align*}
	P[B|A] &= \frac{P[B \cap A]}{P[A]}
\end{align*}

\end{method}

\begin{thm}{Multiplikationsregel}
	Seien $A_1,\ldots A_n$ Ereignisse mit $P[A_i]>0$ (div by $0$ Problem), dann gilt:
	
	\begin{align*}
		&P[A_1 \cap A_2 \cap \cdots \cap A_n] = \\
		&P[A_1] \cdot P[A_2 | A_1] \cdot P[A_3|A_1 \cap A_2] \cdot \ldots \cdot P[A_n | A_1 \cap \cdots \cap A_{n-1}]
	\end{align*}
\end{thm}

\begin{thm}{Satz der totalen Wahrscheinlichkeit}
	 Seien $B_1, \ldots B_n$ eine Zerlegung von $\Omega$ (d.h. $\bigcup_{i=1}^n B_i = \Omega$ und $B_i \cap B_j = \varnothing$ für $i\neq j$), so gilt für ein beliebiges Ereignis $A$:
	 
	 \begin{align*}
	 	P[A] &= \sum_{i=1}^{n} P[A\cap B_i] = \sum_{i=1}^{n} P[A|B_i]\cdot P[B_i]
	 \end{align*}
	 
	 Insbesondere folgt daraus:
	 
	 \begin{align*}
	 	P[A] &= P[A\cap B] + P[A \cap B^c] = P[A|B]\cdot P[B] + P[A|B^c]\cdot P[B^c]
	 \end{align*}
\end{thm}


\begin{thm}{Bayes Theorem}
	Wenn $P[A],P[B].P[B^c]> 0$ so folgt:
	
	\begin{align*}
		P[B|A] = \frac{P[A\cap B]}{P[A]} &= \frac{P[A|B]\cdot P[B]}{P[A|B]\cdot P[B] + P[A|B^c]\cdot P[B^c]}
	\end{align*}
	
	respektive wenn $B_1, \ldots B_n$ eine Zerlegung von $\Omega$ mit $P[B_i]>0 \  \forall i$, dann gilt für ein Ereignis $A$ mit $P[A]>0$:
	
	\begin{align*}
		P[B_k|A] = \frac{P[A\cap B_k]}{P[A]} &= \frac{P[A|B_k]\cdot P[B_k]}{\sum_{i=1}^{n} P[A|B_i] \cdot P[B_i] }
	\end{align*}
\end{thm}


\subsection*{Unabhängigkeit}

\begin{method}{Unabhängigkeit}
	Wir nennen zwei Ereignisse $A$, $B$ unabhängig falls
	
	\begin{align*}
		P[A\cap B] &= P[A] \cdot P[B]
	\end{align*}
	
	Für $P[A] \neq 0$ gilt:
	
	\begin{align*}
		\text{$A,B$ unabhängig} &\iff P[B|A] = P[B]
	\end{align*}
	
	
\end{method}

\begin{method}{Stochastische Unabhängigkeit}
	Wir nennen $A_1,\cdots A_n$ (stochastisch) unabhängig, falls für alle Kombinationen von $A_i, \ldots A_j$ gilt dass
	
	\begin{align*}
	P\left[\bigcap_{i=1}^{m} A_{k_i} \right] = \prod_{i=1}^{m}P[A_{k_i}]
	\end{align*}
	
\end{method}


\section{Diskrete Zufallsvariablen und Verteilungen}



\begin{method}{Diskrete Zufallsvariable (ZV)}
	Sei $(\Omega, \F, P)$ ein diskreter Wahrscheinlichkeitsraum.
	
	Wir nennen 
	$$X:\Omega \to \W(X) = \{x_1,\ldots x_n\}\subseteq \R$$
	eine Zufallsvariable. mit $\W(X)$ der Wertebereich.
	 
	Zusätzlich definieren wir die \textit{Gewichtsfunktion} oder \textit{diskrete Dichte} von $X$ als 
	$$p_X(x_k) = P[X = x_k] = P[\{ \omega | X(\omega) = x_k \}]$$
	sowie auch die Verteilfunktion 
	$$F_x(t) = P[X\leq t] = P[\{\omega | X(\omega) \leq t\}]$$

\end{method}

\subsection*{Verteilungen mehrerer Variablen}


\begin{method}{Gemeinsame Verteilfunktion}
	Seien $X_1,\ldots X_n$ Zufallsvariablen. Die gemeinsame Verteilfunktion $F:\R^n \to [0,1]$ ist definiert durch
	
	\begin{align*}
		(x_1,\ldots x_n)\mapsto F(x_1,\ldots x_n) &= P[X_1 \leq x_1, \ldots X_n \leq x_n]\\
		&= \sum_{y_1 \leq x_1, \ldots, y_n \leq x_n} p(y_1,\ldots, y_n)
	\end{align*}
	
	
\end{method}

\begin{method}{Gemeinsame Verteilfunktion}
	Seien $X,Y$ ZV, so gilt für $F_X : \R \to [0,1]$:
	
	\begin{align*}
		F_X(x) &= P[X \leq x] = P[X \leq x , Y < \infty] = \lim\limits_{y\to\infty} F(x,y)
	\end{align*}
\end{method}

\begin{method}{Gewichtsfunktion der Randverteilung}
	Wir definieren $p_X(x) : \W(X) \to [0,1]$ als die Gewichtsfunktion der Randverteilung von $X$ gegeben durch 

	\begin{align*}
p_X(x) &= P[X=x] = \sum_{y_i\in \W(Y)} p(x,y_i)
	\end{align*}

\end{method}


\begin{method}{Unabhängigkeit von Zufallsvariable}
	Die ZV $X_1, \ldots X_n$ heissen unabhängig, gdw 
	
	\begin{align*}
		F(x_1, \ldots, x_n) &= F_{X_1}(x_1) \cdots F_{X_n}(x_n) \quad \forall x_1,\ldots,x_n\\
		&\text{ oder analog dazu}\\
		p(x_1,\ldots, x_n) &= p_{X_1}(x_1)\cdots p_{X_1n}(x_n) \quad \forall x_1,\ldots,x_n
	\end{align*}

	
	
	
\end{method}



\subsection*{Bedingte Verteilungen}

\begin{method}{Bedingte Verteilungen}
	Seien $X$ und $Y$ Zufallsvariablen mit gemeinsamer Gewichtsfunktion $p(x,y)$. Die bedingte Gewichtsfunktion von $X$ gegeben, dass $Y = y$ ist definiert durch
	
	\begin{align*}
		p_{X|Y} (x|y) := P[X=x | Y=y] = \frac{P[X=x, Y=y]}{P[Y=y]} = \frac{p(x,y)}{p_Y(y)}
	\end{align*}
\end{method}


\begin{method}{Bedingte Erwartung}
	Die Erwartung von $Y$ gegeben, dass $X=x$ bereits eingetroffen ist, ist definiert durch:
	\begin{align*}
	\E[Y|X=x] &= \sum _{y}yP(Y=y\mid X=x)=\sum _{y}y{\frac {P(X=x,Y=y)}{P(X=x)}}\\
	\E[Y|X=x] &= \int y\cdot  f_{Y|X=x}(y)\  dy = \int y\cdot  \frac{f_{X,Y}(x,y)}{f_{X=x}(y)} \ dy
	\end{align*}
\end{method}



\section{Wichige diskrete Verteilungen}

\subsubsection*{Diskrete Gleichverteilung}
Die diskrete Gleichverteilung auf  $\W(X)=\{x_1,\ldots,x_n\}$ hat folgende Eigenschaften:

\begin{align*}
	p_X(x_k) &= P[X = x_k] = \frac{1}{N}
\end{align*}

mit 

\begin{align*}
	\operatorname {E}[X]&={\frac  {1}{n}}\sum _{{i=1}}^{n}x_{i} & \operatorname {Var}[X]&={\frac  {1}{n}}\left(\sum _{{i=1}}^{n}x_{i}^{2}-{\frac  {1}{n}}\left(\sum _{{i=1}}^{n}x_{i}\right)^{2}\right)
\end{align*}

\subsubsection*{Bernoulli Verteilung}
Wir machen ein einziges 0-1 Experiment mit Erfolgswahrscheinlichkeit $p$. Es gilt demnach $\W(X) = \{0,1\}$ mit 

\begin{align*}
	p_X(1) &= P[X=1] = p & 	p_X(0) &= P[X=0] = 1-p
\end{align*}

Woraus folgt:

\begin{align*}
	\E[X] &= p & \Var[X] &= p(1-p)
\end{align*}
Wir schreiben dabei $X\sim {Be}(p)$

\subsubsection*{Binomialverteilung}
Wir möchten gerne die Anzahl Erfolge bei $n$ unabhängigen 0-1 Experimenten mit Erfolgsparameter $p$ beschreiben. Wir schreiben $X \sim \operatorname{Bin}(n,p)$. Dabei ist $\W(X) = \{0,1,\ldots,n\}$ und 

\begin{align*}
	p_X(k) &= P[X=k] = {n \choose k} p^k (1-p)^{n-k}
\end{align*}

sowie 

\begin{align*}
	\E[X] &= np & \Var[X] &= np(1-p)
\end{align*}


\subsubsection*{Geometrische Verteilung}
Wir betrachten eine unendliche Folge von unabhängigen 0-1 Experimenten mit Erfolgsparameter $p$. $X$ sei die Wartezeit auf den ersten Erfolg. Wir schreiben $X\sim\operatorname{Geom}(p)$ und haben:

\begin{align*}
	p_X(k) &= P[X=k] = p(1-p)^{k-1}
\end{align*}
\begin{align*}
	\E[X] &= \frac{1}{p} & \Var[X] &= \frac{1-p}{p^2}
\end{align*}


\subsubsection*{Negativbinomiale Verteilung}
Wir betrachten eine unendliche Folge von unabhängigen 0-1 Experimenten mit Erfolgsparameter $p$. So sei $X$ die Wartezeit auf den $r$-ten Erfolg. Wir schreiben $X\sim \operatorname{NB(r,p)}$.

\begin{align*}
	P_X(k) &= P[X = k] = {k-1 \choose r-1} p^r(1-p)^{k-r}
\end{align*}
\begin{align*}
\E[X] &= \frac{r}{p} & \Var[X] &= \frac{r(1-p)}{p^2}
\end{align*}

\subsubsection*{Hypergeometrische Verteilung}
In einer Urne seien $n$ Gegenstände, davon $r$ vom Typ 1 und $n-r$ vom Typ 2. Man
zieht ohne Zurücklegen $m$ der Gegenstände; die Zufallsvariable $X$ beschreibe die Anzahl
der Gegenstände vom Typ 1 in dieser Stichprobe vom Umfang $m$. Dann hat $X$ eine
hypergeometrische Verteilung mit Parametern $n,m,r$ mit $\W(X) = \{0, 1, \ldots, \min(m, r)\}$ sowie

\begin{align*}
	p_X(k) &= \frac{{r\choose k} {n-r \choose m-k}}{{n \choose m}}
\end{align*}

\begin{align*}
\E[X] &= m\frac{r}{n} & \Var[X] &= m\frac{r}{n}\left(1-\frac{r}{n}\right) \frac{n-m}{n-1}
\end{align*}



\subsubsection*{Poissonverteilung}

Mit Parameter $\lambda>0$ auf $\W(X) = \N_0$ schreiben wir $X\sim {\operatorname{Pois}}(\lambda)$

\begin{align*}
	p_X(k) &= P[X = k] = e^{-\lambda} \frac{\lambda^k}{k!}
\end{align*}

\begin{align*}
\E[X] &= \lambda & \Var[X] &= \lambda
\end{align*}




\section{Allgemeine Zufallsvariable}



\begin{method}{Verteilungsfunktion}
	Wir nennen $F_X : \R \to [0,1]$ mit 
	
	\begin{align*}
		F_X (t) = P[X\leq t]:= P[\{\omega | X(\omega) \leq t\}]
	\end{align*}
	eine Verteilfunktion. Dies hat folgende Eigenschaften:
	\begin{itemize}
		\item $F_X$ ist monoton wachsend $F_X(s)\leq F_X(t)$ für $s\leq t$ und rechtsstetig $F_X(u) \to F_X(t)$ für $u \to t$.
		\item $\lim\limits_{t\to-\infty} F_X(t) = 0$, $\lim\limits_{t\to + \infty} F_X(t) = 1$ 
	\end{itemize}
	umgekehrt ist jede Funktion mit diesen Eigenschaften eine Verteilungsfunktion $F_X$ einer Zufallsvariablen $X$.
\end{method}


\begin{method}{Wahrscheinlichkeitsmass}
	Wir nennen $\mu_X(B) := P[X \in B]$ das Wahrscheinlichkeitsmass mit $\mu_X((-\infty,t]) = F_X(t)$
\end{method}

\begin{method}{Dichtefunktion}
	Falls $F_X(t) = \int_{-\infty}^t f_X(s) ds$ für $\forall t\in\R$ so nennt man $F_X(t)$ absolut stetig und $f_X(s)$ die Dichtefunktion. Mit folgenden Eigenschaften:
	\begin{itemize}
		\item $f_X \geq 0$ und $f_X = 0$ ausserhalb von $\W(X)$.
		\item $\int_{-\infty}^\infty f_X(s) ds = 1$ folgt aus $\lim_{t \to -\infty}F_X(t) = 1$
	\end{itemize}

Umgekehrt kann man aus einer messbaren Funktion $f:\R\to[0,\infty)$ mit $\int_{-\infty}^\infty f(s)ds = 1$ eine Zufallsfariable $X$ konstruieren.
\end{method}



\subsection*{Gemeinsame Verteilungen, unabhängige Zufallsvariablen}


\begin{method}{Gemeinsame Verteilungsfunktion}
	Die gemeinsame Verteilfunktion von $n$ Zufallsvariablen $X_1,\ldots, X_n$ ist die Abbildung $F:\R^n \to [0,1]$ mit
	
	\begin{align*}
	F(x_1,\ldots,x_n) &:= P[X_1 \leq x_1, \ldots , X_n \leq x_n]\\
	F(x_1,\ldots,x_n) &:= \int_{-\infty}^{x_1}\cdots \int_{-\infty}^{x_n}f(t_1,\ldots,t_n)dt_n\ldots dt_1
	\end{align*}
	
	dabei ist $f:\R^n \to [0,\infty)$ die gemeinsame Dichte für welche gilt:
	\begin{itemize}
		\item $f(x_1,\ldots,x_n)\geq 0$ und $=0$ ausserhalb von $\W(X_1,\ldots,X_n)$
		\item $\int_{-\infty}^{\infty}\cdots \int_{-\infty}^{\infty}f(t_1,\ldots,t_n)dt_n\ldots dt_1 = 1$
		\item $P[(X_1,\ldots X_n)\in A] = \int_{(x_1,\ldots,x_n)\in A}f(t_1,\ldots,t_n)dt_n\ldots dt_1$
	\end{itemize} 
	TODO: muss ich das dritte Prüfen?
\end{method}

\begin{method}{Randverteilung} Haben $X,Y$ die gemeinsame Dichtefunktion $F$, so ist die Funktion $F_X : \R \to [0,1]$ mit 
	
	\begin{align*}
	F_X(x) := P[Y \leq y] = P[x\leq Y, Y < \infty] = \lim_{y \to \infty} F(x,y)
	\end{align*}
	die sogenannte Randverteilung von $X$.
	
	Falls die Dichte $f(x,y)$ existiert so gilt:
	
	\begin{align*}
	f_X(x) &= \int_{-\infty}^{\infty}f(x,y) dy\\
	f_X(x) &= \frac{d}{dx}F_X(x) = \frac{d}{dx}\lim_{y \to \infty} F(x,y)
	\end{align*}
	
\end{method}

\textbf{Beispiel zusammengesetzter Zufallsvariablen}

\begin{itemize}
	\item $X = \min\{X_1,\ldots,X_n\}$, finde $f_X$
	
	Wir wissen $F_X(t) = P[X\leq t] = \prod_{i=1}^n P[X_k \leq t] = F(t)^n$. Da $f_X(t) = \frac{dF_X(t)}{dt}$ folgt $f_X(t) = n\cdot F(t)^{n-1} f(t)$.
	
	\item $X = \max\{X_1,\ldots,X_n\}$, finde $f_X$
	
	Wir wissen $F_X(t) = 1 - P[X>t] = 1 - P [X_1 > t, \ldots X_n > t] = 1 - (1 - F(t))^n$. Da $f_X(t) = \frac{dF_X(t)}{dt}$ folgt $f_X(t) = n(1-F(t))^{n-1}f(t)$.
\end{itemize}




\begin{method}{Unabhängigkeit von Zufallsvariablen} Die Zufallsvariablen $X_1,\ldots, X_n$ heissen unabhängig, falls gilt
	
	\begin{align*}
	F(x_1,\ldots x_n) &= F_{X_1}(x_1) \cdots F_{X_n}(x_n) \quad \text{respektive:}\\
	f(x_1,\ldots x_n) &= f_{X_1}(x_1) \cdots f_{X_n}(x_n) \quad \forall x_1,\ldots x_n
	\end{align*}
\end{method}


\subsection*{Transformation von stetigen Zufallsvariablen}
\begin{itemize}
	\item Angenommen wir haben die Dichte $f(x,y)$ gegeben und wir suchen die Verteilfunktion und Dichte von $Z = X + Y$. Wir haben $F_Z(z) = P[Z \leq z] = P[X+Y\leq z]$. Nun definieren wir $A_z = \{(x,y)\in\R^2|x+y\leq z\}$. Es gilt nun:
	
	\begin{align*}
	F_Z(z) = P[(X,Y)\in A_z] &= \int_{A_z} \int f(x,y) dy dx\\
	&= \int_{-\infty}^{\infty} \int_{-\infty}^{z-x} f(x,y) dy dx
	\end{align*}
	
	Sowie $f_Z(z) = \frac{d}{dz} F_Z(z)$.
	
	
\item Sei $U = \frac{X}{X+Y}$ mit $X,Y\sim Exp(\lambda)$; finde $f_U, F_U$:

Es gilt $F_U(u) = P[U \leq u] = P[\frac{X}{X+Y}\leq u]$ woraus $x(u^{-1} -1)\leq y$ folgt.
Somit ist das Integral $F_U = \lambda^2 \int_{0}^{\infty} e^{-\lambda x} \int_{x(u^{-1} -1)}^{\infty} e^{-\lambda y} dy dx = u$.

Da nun $0 \leq U = \frac{X}{X+Y} \leq 1$ folgt $f_U(u) = 1_{u\in[0,1]}$

\item Sei nun zusätzlich noch $V=X+Y$ und wir möchten $f_{U,V}$ berechnen, so gilt

$$f_{U,V} = \lambda^2 \int_{0}^{\infty} e^{-\lambda x} \left(\int_{0}^{\infty} 1_{\{\frac{x}{x+y}\leq u, x+y \leq v\}} e^{-\lambda y}dy \right) dx$$

\end{itemize}





\subsection*{Wichtige stetige Verteilungen}

\subsubsection*{Gleichverteilung}
Die Gleichverteilung auf $[a,b]$ mit $\W(X)=[a,b]$, genannt $X\sim\U(a,b)$:

\begin{align*}
	f_X(t) &= \begin{cases}
	\frac{1}{b-a} \ \text{ für } a \leq t \leq b\\
	0 \ \text{ sonst}
	\end{cases}
	& 
	F_X(t) &= \begin{cases}
	0 \ \text{ für } t<a \\
	\frac{t-a}{b-a} \ \text{ für } a \leq t \leq b\\
	1 \ \text{ für } t > b
	\end{cases}
\end{align*}


\begin{align*}
\E[X] &= {\displaystyle\int \limits _{-\infty }^{\infty }xf(x)\,dx={\frac {1}{b-a}}\int \limits _{a}^{b}x\cdot 1\,dx={\frac {1}{2}}{\frac {b^{2}-a^{2}}{b-a}}={\frac {a+b}{2}}} \\ \Var[X] &={\frac {1}{b-a}}\int \limits _{a}^{b}{x^{2}\cdot 1\,dx}-\left({\frac {a+b}{2}}\right)^{2}={\frac {1}{3}}{\frac {b^{3}-a^{3}}{b-a}}-\left({\frac {a+b}{2}}\right)^{2}
\end{align*}




\subsubsection*{Exponentialverteilung}
Die Exponentialverteilung mit Parameter $\lambda > 0$ mit $\W(X)=[0,\infty)$, genannt $X\sim Exp(\lambda)$:

\begin{align*}
f_X(t) &= \begin{cases}
\lambda \cdot e^{-\lambda t} \ \text{ für } t \geq 0\\
0 \text{ für } t < 0
\end{cases}
& 
F_X(t) &= \begin{cases}
1-e^{-\lambda t} \text{ für } t\geq 0\\
0 \ \text{ für } t < 0
\end{cases}
\end{align*}

\begin{align*}
	\E[X]&=\int \limits _{0}^{\infty }\lambda x{\mathrm  {e}}^{{-\lambda x}}\,{\mathrm  {d}}x={\frac  {1}{\lambda }}\\
	\Var[X] &= \frac{1}{\lambda^2}
\end{align*}

Die Exponentialverteilung ist Gedächtnislos: 
$$P[X > t+s| X > s] = P[X > t]$$

\textbf{Beispiel (Casinogewinn)} Wenn $Z\sim Exp(\lambda)$ dürfen wir $c$ wählen so dass wenn $Z > c$ gewinnen wir $c$. Berechne maximal erwartete Auszahlung: $G(c) = \E[c1_{\{Z > c\}}] = cP[Z > c] = ce^{-\lambda c}$. Nun maximiere $c$.

\subsubsection*{Normalverteilung}
Die Normalverteilung mit Parameter $\sigma^2$, der Varianz, $\mu$, dem Erwartungswert sowie $\W(X) = \R$. Man schreibt $X\sim \Nor(\mu,\sigma^2)$

\begin{align*}
f_X(t) &= \frac{1}{\sigma\sqrt{2\pi}}e^{-\frac{(t-\mu)^2}{2\sigma^2}}
\end{align*}

Problem: Das Integral ist mühsam. Trick Wir plotten die Dichte $\varphi(t)$ sowie Verteilfunktion $\Phi(t)$ von $\Nor(0,1)$ und benutzen dass $\frac{X-\mu}{\sigma} \sim \Nor(0,1)$.

Es gilt $\Phi(-x) = 1 - \Phi(x)$ sowie ${\Phi^{-1}(x) = -\Phi^{-1}(1-x)}$


Zudem gilt für $X\sim N(\mu_X,\sigma^2_X)$, $Y \sim N(\mu_Y, \sigma^2_Y)$ mit $Z=X+Y$ dass $Z \sim N(\mu_X+\mu_Y, \sigma^2_X + \sigma^2_Y)$.


\subsubsection*{Paretoverteilung}
Die Paretoverteilte Variable $X\sim\operatorname{Par}(x_0,\alpha)$ mit 

\begin{align*}
f_X(t) &= \begin{cases}
\frac{\alpha x_0 ^\alpha}{x^{\alpha+1}} \ \text{ für } x \geq x_0 \\
0 \text{ sonst}
\end{cases}
& 
F_X(t) &= \begin{cases}
1-\left(\frac{x}{x_0}\right)^{-\alpha} \text{ für } x \geq x_0\\
0 \ \text{ sonst}
\end{cases}
\end{align*}

\begin{align*}
\E[X] &= \begin{cases}
x_0 \frac{\alpha}{\alpha - 1}\ \text{ für } \alpha > 1 \\
\infty \ \text{ sonst}
\end{cases}
& 
\Var[X] &= \begin{cases}
\frac{x_0 ^2 \alpha}{(\alpha-1)^2(\alpha-2)} \text{ für } \alpha > 2\\
\infty \ \text{ für } 1 < \alpha \leq 2
\end{cases}
\end{align*}

Diese Verteilung modelliert zum Beispiel die Krankenkosten eines Versicherten pro Jahr.


\section{Kennzahlen von Zufallsvariablen}


\begin{method}{Erwartungswert}
	\begin{itemize}
		\item Diskret
		\begin{align*}
	\mu = \E[X] &= \sum_{x_i \in \mathcal{W}(X)} x_i \cdot P(X=x_i) = \sum_{x_i \in \mathcal{W}(X)} x_i \cdot p(x_i) \\
	\E(g(X)) &= \sum_{x_i \in \mathcal{W}(X)} g(x_i) \cdot P(X=x_i)
	\end{align*}
		\item Stetig:
\begin{align*}
\mu = \E[X] &= \int_{-\infty}^{\infty} x \cdot f_X(x) \; \rmd x
\end{align*}
		
		\item Allgemein:
		\begin{align*}
	\E(aX+b) &= a\E[X] + b\\
	\E \left( \sum_{i=1}^n X_i \right) &= \sum_{i=1}^n \E(X_i)
		\end{align*}	
		Sind $X_i$, $X$ und $Y$ unabhängig, so gilt
		\begin{align*}
		\E(XY) &= \E[X]\E(Y)\\
		\E \left( \prod X_i \right) &= \prod \E(X_i)\\
		\end{align*}
		
	\end{itemize}
	

\end{method}


\begin{method}{Varianz}
\begin{itemize}
\item Diskret
		\begin{align*}
		\Var[X] &= \sum_{x_i \in \mathcal{W}(X)} (x_i - \mu)^2 \cdot P(X=x_i) \\
\Var[X] &= \E((X - \mu)^2) = \E(X^2) - \E[X]^2
	\end{align*}
\item Stetig:
	\begin{align*}
\Var[X] =  \E{[X^2]} - \E{[X]}^2  &= \int_{-\infty}^{\infty} (x - \E[X])^2 \cdot f_X(x) \; \rmd x
\end{align*}

\item Allgemein:
\begin{align*}
	\Var(aX+b) &= a^2 \cdot \Var[X]\\
	\Var[X] &= E(X^2)-E[X]^2\\
	\Var[X] &= E((X-\mu)^2)\\
	\Var(aX+bY+c) &= a^2\Var[X] +b^2\Var(Y) + 2 ab \operatorname{Cov}(X,Y)
\end{align*}


Für unkorrelierte Zufallsvariablen $X_i$ gilt
\begin{align*}
\Var \left( \sum_{i=1}^n X_i \right) &= \sum_{i=1}^n \Var(X_i)\\
\Var (\overline X_n) &= \frac{\sigma^2}{n}
\end{align*}


	
\end{itemize}
\end{method}

\TODO{serie 9: transformation in varianz}

\textbf{Beispiel (nicht konvergenter Erwartungswert):}
Sei $X$ eine Cauchy-verteilte Zufallsvariable, mit $\W(X) = \R$ und

\begin{align*}
f_X(t) &= \frac{1}{\pi} \frac{1}{1+t^2}
& 
F_X(t) &= \frac{1}{2} + \frac{1}{\pi}\arctan(t)
\end{align*}

Dann gilt:

\begin{align*}
\int_{-\infty}^\infty |x|f_X(x) dx &= \lim\limits_{b\to\infty} \frac{1}{\pi}\log(1+b^2) = +\infty
\end{align*}





\begin{method}{Standardabweichung}
\begin{align*}
&\sigma = \sqrt{\Var[X]} \\
&\sigma_{aX+b} = |a|\cdot \sigma_X
\end{align*}
\end{method}


\begin{method}{$\alpha$-Quantile, Median}
\begin{align*}
\text{Sei} \quad& 0<\alpha<1   \\
P(X \leq q(a))=\alpha \qquad &\implies \qquad F_X(q(a))=\alpha \\
\phantom{\quad P(X \leq q(a))=\alpha \qquad}& \implies \qquad q(a)=F_X^{-1}(a)\\
\text{Median:} &\quad \tfrac{1}{2}\text{-Quantil} 
\end{align*}
\end{method}

\begin{method}{{Lineare Transformation}}
\begin{align*}
Y = aX+b \quad &\implies \quad f_Y(x) = \tfrac{1}{|a|} f_X\left(\tfrac{x-b}{|a|}\right)
\end{align*}
\end{method}




\subsection*{Kovarianz und Korrelation}


\begin{method}{Kovarianz}
	\begin{align*}	
	\operatorname{Cov}(X,Y) &= \E\left(\left(X-\E[X]\right)\left(Y-E(Y)\right)\right) \\ &= \E[XY]  - \E[X]\E[Y]\\
	\operatorname{Cov}(X,X) &= \Var(X)
	\end{align*}
\end{method}

\begin{method}{Korrelation}
	\begin{align*}
	\operatorname{Corr}(X,Y) &= \rho_{XY} = \frac{\operatorname{Cov}(X,Y)}{\sqrt{\Var(X)} \cdot \sqrt{\Var(Y)}}
	= \frac{\operatorname{Cov}(X,Y)}{\sigma_X\cdot\sigma_Y} \\
	\rho_{XY} \in [-1,1] &  \text{\small als dimensionsloses Mass des linearen Zusammenhanges} \\
	\rho_{XY} &=\pm 1 \quad \iff \quad Y = a\pm bX \quad \text{mit }b>0, a \in \R
	\end{align*}
\end{method}

\begin{thm}{Rechenregeln}
	\begin{align*}
	\operatorname{Cov}(X,Y) &= \E(X\cdot Y) - \E[X] \cdot \E(Y) \\
	\operatorname{Cov}(\cdot,\cdot) &\text{ ist bilinear:} \quad \\
	\operatorname{Cov}(a+bX,c+dY) &= b\cdot d \cdot \operatorname{Cov} (X,Y) \\
	\operatorname{Cov}(X+Y,Z) &= \operatorname{Cov}(X,Z)+\operatorname{Cov}(Y,Z) \\
	\Var(X+Y) &= \Var(X)+\Var(Y)+2 \cdot \operatorname{Cov}(X,Y) \\
	\operatorname{Cov} &\text{ ist symmetrisch}
	\end{align*}
	Sind die Zufallsvariablen $X$ und $Y$ unabhängig, so gilt $\operatorname{Cov}(X,Y) = \operatorname{Corr}(X,Y) = 0$.
	Die Umkehrung gilt jedoch im Allgemeinen nicht.
\end{thm}



\begin{mdframed}
	\centering
	{unabhängig $\implies$ paarweise unabhängig $\implies$ unkorreliert}
\end{mdframed}


\begin{thm}{Transformation von Zufallsvariablen}
	Sei $X$ eine Zufallsvariable und $Y=g(X)$ eine weitere. Ist $X$ stetig mit Dichte $f_X(x)$ so ist:
	\begin{align*}
		\E[Y] = \E[g(X)] = \int_{-\infty}^{+\infty} g(x) f_X(x) dx
	\end{align*}
	
	Falls $X,Y$ zwei stetige Zufallsvariablen sind $g(x,y) : \R^2 \to \R$ eine Funktion, dann gilt:
	
	\begin{align*}
		\E[g(X,Y)] &= \int_{-\infty}^{+\infty} \int_{-\infty}^{+\infty} g(x,y) f_{X,Y}(x,y) dx dy
	\end{align*}
	
\end{thm}




\subsection*{Mehrere i.i.d Zufallsvariablen}

Seien $X_1,\ldots,X_n \overset{iid}{\sim} F$. Dann ist $Y=g(X_1,\ldots,X_n)$ mit $g: \R^n \longrightarrow \R$ eine neue Zufallsvariable.
Wichtige Vertreter sind die Summe von Zufallsvariablen und die relative Häufigkeit.
\[ S_n = X_1 + \ldots + X_n \qquad \qquad\overline{X}_n = \dfrac{1}{n} S_n \]
Die Verteilungen von $S_n$ und $\overline X_n$ sind im allgemeinen aber schwierig zu bestimmen.

\subsubsection*{Bekannte Verteilungen}
\begin{align*}
%X_i \overset{iid}{\sim} \operatorname{Exponential}(\lambda) \; \Rightarrow & \; S_n \sim \operatorname{Exponential}(n\cdot \lambda) \\
X_i \overset{iid}{\sim} \operatorname{Poisson}(\lambda_i) \; \Rightarrow & \; S_n \sim \operatorname{Poisson}({\textstyle \sum \lambda_i}) \\
X_i \overset{iid}{\sim} \operatorname{Bernoulli}(p) \; \Rightarrow & \; S_n \sim \operatorname{Binomial}(n,p)\\
X_i \overset{iid}{\sim} \mc N(\mu, \sigma^2) \; \Rightarrow & \; S_n \sim \mc N(n \cdot \mu, n \cdot \sigma^2)
\end{align*}
\subsubsection*{Erwartungswerte und Varianzen}
\begin{align*}
\E(S_n) &= n \cdot \E(X_1) & \E(\overline{X}_n) &= \E(X_1) \\
\Var(S_n) &= n \cdot \Var(X_1) & \Var(\overline{X}_n) &= \tfrac{1}{n} \Var(X_1) \\
\sigma_{S_n} &= \sqrt{n} \cdot \sigma_{X_1} & \sigma_{\overline{X}_n} &= \tfrac{1}{\sqrt{n}} \cdot \sigma_{X_1}
\end{align*}



\section{Ungleichungen und Grenzwertsätze}

Wir bezeichnen in der folgenden Diskussion $X_i, \ldots, X_n$ Zufallsvariablen. (welche meistens i.i.d sind). Dann definieren wir

\begin{align*}
	S_n &= \sum_{i=1}^{n} X_i & \overline{X_n} &= \frac{1}{n}S_n = \frac{1}{n}\sum_{i=1}^{n}X_i
\end{align*}

\textbf{i.i.d. Zufallsvariablen} Dies sind Zufallsvariablen die unabhängig und identisch verteilt sind. 

\subsection*{Allgemeine Ungleichungen}

\begin{method}{Markov Ungleichung}
	Sei $X$ eine Zufallsvariable und $g:W(X)\to [0,\infty)$ eine wachsende Funktion. Für jedes $c\in\R$ gilt dann 
	
	\begin{align*}
		P[X \geq c] &\leq \frac{\E[g(X)]}{g(c)}
	\end{align*}
\end{method}


\begin{method}{Chebyshev Ungleichung}
	Sei $X$ eine Zufallsvariable mit endlicher Varianz. Für jedes $b>0$ gilt:
	
	\begin{align*}
		P[|X-\E[X]| \geq b] &\leq \frac{\Var[X]}{b^2}
	\end{align*}
\end{method}


\begin{method}{Schwache Gesetz der grossen Zahlen}
	Sei $X_1,X_2,\ldots$ eine Folge von \textit{unabhängigen} Zufallsvariablen mit gleichem Erwartungswert $\E[X_i] = \mu$ sowie gleicher Varianz $\Var[X_i] = \sigma^2$.\\
	
	
	Sei $\overline{X}_n = \frac{1}{n}S_n = \frac{1}{n} \sum_{i=1}^{n} X_i$. Dann konvergiert $\overline{X}_n$ für $n\to\infty $ stochastisch gegen $\mu = \E[X_i]$, respektive:
	
	\begin{align*}
		P[|\overline{X}_n - \mu|> \epsilon] \stackrel{n\to\infty}{\longrightarrow}   0 \ \text{ für jedes }\epsilon > 0
	\end{align*}
	
	\textit{Bemerkung:} Es reicht wenn die $X_i$ nur paarweise unkorreliert sind (d.h. $\operatorname{Cov}(X_i,X_k) = 0$ für $i \neq k$)
	
\end{method}


\begin{method}{Starke Gesetz der grossen Zahlen}
	Sei $X_1,X_2,\ldots$ eine Folge von \textit{unabhängigen} Zufallsvariablen die alle die gleiche Verteilung haben mit Erwartungswert $\E[X_i] = \mu$ endlich. 
	
	Für $\overline{X}_n = \frac{1}{n}S_n = \frac{1}{n} \sum_{i=1}^{n} X_i$ gilt dann 
	
	\begin{align*}
		\overline{X_n} \stackrel{n \to \infty}{\longrightarrow} \mu \quad \text{P-fastsicher}.
	\end{align*}  
	respektive
	\begin{align*}
	P\left[ \{ \omega \in \Omega | \overline{X_n}(\omega)  \stackrel{n \to \infty}{\longrightarrow}  \mu  \} \right] &= 1
	\end{align*}
\end{method}

\textbf{Unterschied zwischen starkem und schwachem Gesetz der grossen Zahlen:} Beim ersten Gesetz ist die Wahrscheinlichkeit, dass $|\overline{X}_n - \mu|> \epsilon$ nie 0 sondern nur asymptotisch mit $n \to \infty$ 0. Somit gibt es eine sehr kleine Wahrscheinlichkeit, dass $|\overline{X}_n - \mu| > \epsilon$. Beim starken Gesetz ist jedoch die Wahrscheinlichkeit, dass dies passiert 0. In anderen Worten wenn wir unendlich viele $\overline{X_n}$ anschauen werden nur endlich viele nicht nach $\mu$ konvergieren.

Zusammengefasst: Beim schwachen ist die Chance von Nichtkonvergenz sehr klein, beim zweiten 0.

\subsection*{Der zentrale Grenzwertsatz}


\begin{method}{Zentraler Grenzwertsatz}
	Sei $X_1,X_2,\ldots$ eine Folge von \textit{i.i.d Zufallsvariablen (independent and identically distributed random variables)} mit $\E[X_i] = \mu$ und $\Var[X_i] = \sigma^2$. Für die Summe $S_n = \sum_{i=1}^{n} X_i$ gilt dann:
	
	\begin{align*}
		\lim\limits_{n \to \infty} P\left[ \frac{S_n - n\mu}{\sigma \sqrt{n}} \leq x\right] = \Phi(x)
	\end{align*}

	Wobei $\Phi$ die Verteilungsfunktion von $\mathcal{N}(0,1)$ ist.	
\end{method}

In Praxis definiert man $S_n^* = \frac{S_n- n\mu}{\sigma \sqrt{n}} = \frac{S_n-\E[S_n]}{\sqrt{\Var[S_n]}}$ da $\E[S_n] = n\mu$ sowie $\Var[S_n] = n\sigma^2$. Als auch definieren wir $\overline{X_n} = \frac{1}{n}S_n$.

Nun gilt

\begin{align*}
	P[S_n^*\leq x] &\approx \Phi(x) \quad \text{ für $n$ gross}\\
	S_n^* &\stackrel{\text{approx}}{\sim} \mathcal{N}(0,1)\\
	S_n &\stackrel{\text{approx}}{\sim} \mathcal{N}(n\mu, n\sigma^2)\\
	\overline{X_n} &\stackrel{\text{approx}}{\sim} \mathcal{N}(\mu, \frac{1}{n}\sigma^2)
\end{align*}


\begin{thm}{Kontinuitätskorrektur}
	
	
	Angenommen $S_n \stackrel{\text{approx.}}{\sim} \mathcal{N}(np,np(1-p))$ sowie
	\begin{align*}
		P\left[ a < S_n  \leq b \right] &= P\left[ \frac{a-np}{ \sqrt{np(1-p)}} < S_n^* \leq \frac{b-np}{\sqrt{np(1-p)}} \right]\\
		&\approx \Phi \left(\frac{b+\frac{1}{2} - np}{\sqrt{np(1-p)}}\right) - \Phi \left(\frac{a+\frac{1}{2} - np}{\sqrt{np(1-p)}}\right)
	\end{align*}
	wobei $S^*_n \sim \mathcal{N}(0,1)$, so haben wir die Untere Schranke um $\frac{1}{2}$ nach oben verschoben, da $a$ nicht drin ist, und $b$ auch, da $b$ drin ist. Intuitiv wollen wir damit die Stäbe aus dem Histogramm der Binomialverteilung zentriert über die Werte von $a$ bis $b$ setzten.
\end{thm}


\subsection*{Grosse Abweichungen und Chernoff-Schranken}


\begin{method}{Momenterzeugende Funktion}
	Wir definieren die momenterzeugende Funktion einer Zufallsvariable X als
	\begin{align*}
		M_X(t) &= \E\left[e^{tX}\right]
	\end{align*}
	welche auf $[0,\infty)$ wohldefiniert ist aber unendlich gross werden kann.
	
 Aufgrund der Definition gilt:\begin{itemize}
 	\item Diskret: $M_X(t) = \sum_{i=1}^{\infty} e^{tx_i}p_i$
 	\item Stetig: $M_X(t)= \int_{-\infty}^{\infty} e^{tx}f(x)\ dx$
 \end{itemize}
\end{method}

\textbf{Beispiel}
Sei $X \sim Exp(\lambda)$. Dann gilt $M_X(t) = \E[e^{tX}] = \int_{0}^{\infty} e^{tx} \lambda e^{-\lambda x} dx = \lambda \int_{0}^{\infty}e^{-(\lambda-t)x} dx = - \frac{\lambda}{\lambda - t} e^{-(\lambda -t)x}\mid_0^\infty$
\begin{method}{Abschätzung mit momenterzeugender Funktion}
Seien $X_1,\ldots,X_n$ i.i.d. Zufallsvariablen für welche die momenterzeugende Funktion $M_X(t)$ für alle $t\in \R$ endlich ist. Für jedes $b$ gilt dann:
\begin{align*}
	P[S_n > b] &\leq \exp \left( \inf_{t\in\R} \left(  n\log M_X(t) - tb \right)  \right)
\end{align*}
\end{method}


\begin{method}{Chernoff Schranke}
	Seien $X_1,\ldots,X_n$ unabhängig mit $X_i \sim Be(p_i)$ und $S_n = \sum_{i=1}^{n}X_i$. Sei ferner $\mu_n = \E[S_n] = \sum_{i=1}^{n} p_i$ und $\delta > 0$. Dann gilt:
	\begin{align*}
		P[S_n \geq (1+\delta)\mu_n] &\leq \left(\frac{e^\delta}{(1+\delta)^{1+\delta}}\right)^{\mu_n}
	\end{align*}
\end{method}

Alternativ gibt es auch folgende Abschätzungen: (nicht im Skript!)

\begin{align*}
	P[X \geq (1+\delta)\mu] &\leq e^{-\frac{\delta^2}{2+\delta}\mu} \ \text{ für alle } \delta > 0  \\
	P[X \leq (1-\delta)\mu] &\leq e^{-\mu\delta^2/2} \ \text{ für } 0 < \delta < 1
\end{align*}


\part{Statistik}

\section{Schätzer}
\subsection*{Grundbegriffe}

\begin{itemize}
	\item $\vartheta =  (\vartheta_1,\ldots,\vartheta_n)$ ist ein Vektor an Parameter die wir gerne Schätzen möchten. Sie müssen Teil von der Dichtefunktion sein.
	\item $T = (T_1,\ldots,T_n)$ sind Schätzer, das sind Zufallsvariablen die $\vartheta$ bestmöglichst schätzen. Dabei gilt $T_i = t_i(X_1,\ldots,X_n)$ für ein geeignetes $t_i:\R^n \to \R$
	\item Wir nennen $x_1,\ldots,x_n$ Daten/Realisationen von Zufallsvariablen mit $x_i = X_i(\omega)$.
	\item $T_i(\omega) = t_i(X_1(\omega), \ldots, X_n(\omega)) = t_i(x_1,\ldots,x_n) $ heisst Schätzwert.
\end{itemize}

\subsection*{Gütebegriffe}

\begin{method}{Erwartungstreue \& Bias}
	\begin{align*}
	\E_\vartheta[T] = \vartheta &\iff \text{ Erwartungstreu}\\
	bias &= \E_\vartheta[T] - \vartheta
	\end{align*}
\end{method}

\begin{method}{Konsistenz}
	Wir nennen eine Folge von Schätzern $T^{(n)}$ konsistent für $\vartheta$ falls
	\begin{align*}
\lim_{n\to\infty} P_\vartheta [|T^{(n)} - \vartheta| > \epsilon] = 0 \quad \forall \epsilon > 0
	\end{align*}
\end{method}



\begin{method}{Mean Squared Error (MSE)}
\begin{align*}
	MSE_\vartheta[T] &= \E_\vartheta [(T-\vartheta)^2]= \Var_\vartheta[T] + (E_\vartheta [T]-\vartheta)^2\\
	&=  \Var_\vartheta[T] + (bias^2)
\end{align*}
\end{method}




\subsection*{Maximum-Likelihood-Methode}

\begin{method}{Maxmium-Likelihood-Schätzer}
Seien $X_1,\ldots X_n$ Zufallsvariablen von $n$ Stichproben und $x_1,\ldots,x_n$ die Realisationen dieser Zufallsvariablen.	
	
Wir definieren die Likelihood-Funktion als

\begin{align*}
	L(x_1,\ldots x_n;\vartheta) &= \begin{cases}
	p(x_1,\ldots,x_n;\vartheta) \quad \text{ diskreter Fall}\\
	f(x_1,\ldots,x_n;\vartheta) \quad \text{ stetiger Fall}\\
	\end{cases}
\end{align*}

Wenn die $X_1,\ldots,X_n$ i.i.d. sind, so gilt $p(x_1,\ldots,x_n;\vartheta) = \prod_{i=1}^{n}p_X(x_i;\vartheta)$ respektive $f(x_1,\ldots,x_n;\vartheta) = \prod_{i=1}^{n}f_X(x_i;\vartheta)$

Wenn die $X_1,\ldots X_n$ i.i.d. sind, dann verwenden wir oft die log-likelihood-Funktion $\log(L)$ welche das $\prod$ in ein $\sum$ verwandelt.

Nun maximiere die Funktion (meistens reicht die Nullstelle): Die gibt dann z.B. eine Funktion wie $\vartheta = \frac{1}{n}\sum_{i=1}^{n}x_i$, wir ersetzten nun $x_i$ durch $X_i$ und $\vartheta$ durch $T$ zu $T = \frac{1}{n}\sum_{i=1}^{n}X_i$
\end{method}

\textbf{Beispiel 1:}

Seien $X_1,\ldots,X_n$ i.i.d und das Modell $P_{\theta}\sim Poisson(\theta)$. Wir bauen log-Likelihood Funktion:

\begin{align*}
	\log(L(x_1,\ldots,x_n)) &= \log\left(\prod_{i=1}^{n} e^{-\theta} \frac{\theta^{x_i}}{x_i !} \right)\\ &= -\theta n + \log(\theta) \sum_{i=1}^{n} x_i - \sum_{i=1}^{n}\log(x_i !)
\end{align*}

nun leiten wir ab:

\begin{align*}
	\frac{\partial}{\partial \theta}\log(L(x_1,\ldots,x_n)) &= -n + \frac{\sum_{i=1}^{n}x_i}{\theta} \stackrel{!}{=} 0 \\
	\implies & \theta = \frac{\sum_{i=1}^{n} x_i}{n}
\end{align*}

\textbf{Beispiel 2 (nicht Ableitbar):}

Sei $f_X(x) = \begin{cases}
e^{\alpha - x} \quad x \geq \alpha\\
0 \quad \text{sonst}
\end{cases}$ Wir möchten den MLS berechnen:

\begin{align*}
	L(\vec{x},\theta) &= \prod_{i=1}^{n} f_\alpha (x_i) = \prod_{i=1}^{n} e^{\alpha - x_i} 1_{x_i \geq \alpha}\\
	&= \exp \left(n\alpha - \sum_{i=1}^{n} x_i \right) \cdot \prod_{i=1}^{n} 1_{x_i\geq \alpha}\\
	&= \exp \left(n\alpha - \sum_{i=1}^{n} x_i \right) \cdot 1_{\min x_i \geq \alpha}\\
	&= \begin{cases}
	\exp (n \alpha - \sum_{i=1}^{n} x_i) \quad \alpha \leq \min x_i\\
	0 \quad \text{sonst}
	\end{cases}
\end{align*}
Somit ist der MLS $\hat{\alpha} = \min x_i$, da dann $\alpha$ den obigen Ausdruck maximiert.

\subsection*{Momentenschätzer}

\begin{method}{Momentenschätzer}
	
	Die Idee ist, dass wir ein Gleichungssystem auflösen bei welchen:
	\begin{itemize}
\item Das $j$-te Moment ${\displaystyle m_{j}(\vartheta )=\operatorname {E} _{\vartheta }(X_{1}^{j})}$
\item Das $j$-te Stichprobenmoment ${\displaystyle m_{j}(x)={\frac {1}{n}}\sum _{i=1}^{n}x_{i}^{j}}$
	\end{itemize}
	
	Dann setzen wir die Momente gleich und lösen nach $\vartheta$ auf.
	
\end{method}

Es gilt bekanntlicherweise dass $\E[X] = \mu$ sowie $\E[X^2] = \mu^2 + \sigma^2$. Wir können somit das folgende System auflösen:

\begin{align*}
	\frac{x_1 + \ldots + x_n}{n} &= \mu\\
	\frac{x_1^2 + \ldots + x_n^2}{n} &= \mu^2 + \sigma ^2
\end{align*}



\begin{thm}{Zentrale Lemmata}
	Seien $X_1,\ldots, X_n$ i.i.d. $\sim \mathcal{N}(\mu,\sigma^2)$ dann gilt:
	
	\begin{itemize}
		\item $\overline{X_n}$ ist Normalverteilt $\sim \mathcal{N}(\mu,\frac{1}{n}\sigma^2)$ und $\frac{\overline{X_n} - \mu}{\sigma / \sqrt{n}} \sim \mathcal{N}(0,1)$
		\item $\frac{n-1}{\sigma^2}S^2 = \frac{1}{\sigma^2}\sum_{i=1}^{n}(X_i - \overline{X_n})$ ist $\chi^2$ Verteilt mit $n-1$ Freiheitsgraden.
		\item $\overline{X_n}$ und $S^2$ sind unabhängig.
		\item Der Quotient
		\begin{align*}
\frac{\overline{X_n}-\mu}{S/\sqrt{n}} = \frac{\frac{\overline{X_n} - \mu}{\sigma / \sqrt{n}}}{S/\sigma} = \frac{\frac{\overline{X_n} - \mu}{\sigma / \sqrt{n}}}{\sqrt{\frac{1}{n-1} \frac{n-1}{\sigma^2} S^2}}
		\end{align*}
		ist $t$-verteilt mit $n-1$ Freiheitsgraden.
	\end{itemize}
Dabei galt $\overline{X_n} = \frac{1}{n}\sum_{i=1}^{n} X_i$ sowie $S^2 = \frac{1}{n-1} \sum_{i=1}^{n} (X_i -\overline{X_n})^2$ die empirische Stichprobenvarianz.
\end{thm}

\section{Tests}

Für einen Test definieren wir:
\begin{itemize}
	\item Hypothese $H_0$ : $\vartheta \in \Theta_0$
	\item Alternative $H_A$ : $\vartheta \in \Theta_A$
\end{itemize}

Wir bauen eine Teststatistik $T$. Wir verwerfen die Hypothese wenn $T(\omega) \in K$ liegt, wobei $K$ ein Intervall, der Verwerfungsbereich ist.

\begin{itemize}
	\item Signifikanzniveau $\alpha \in (0,1)$ wird gewählt mit $\sup_{\vartheta\in \Theta_0} P_{\vartheta} \left[T\in K\right] \leq \alpha$
	\item Die Macht des Testes ist $\beta : \Theta_A \to [0,1]$ mit $\beta(\vartheta) = P_{\vartheta}[T\in K]$.
\end{itemize}

Zuerst wählt man das Signifikanzniveau dann maximiert man die Macht des Testes.

\TODO{add example}


\begin{thm}{Neyman-Pearson-Lemma}
	Wenn $\Theta_0 = \{\vartheta_0\}$, $\Theta_A = \{\vartheta_A\}$, $K = [0,c)$ sowie $\alpha^*= P_{\vartheta_0}[T\in K] = P_{\vartheta_0}[T < c]$.
	
	Jeder andere Test als 
	\begin{align*}
		R(x_1,\ldots,x_n;\vartheta_0, \vartheta_A) = \frac{L(x_1,\ldots,x_n;\vartheta_0)}{L(x_1,\ldots,x_n;\vartheta_A)}
	\end{align*}der Likelihood Koeffizient, mit Signifikanzniveau $\alpha \leq \alpha'$ hat kleinere Macht.
\end{thm}


\subsection*{Statistische Testverfahren}


\begin{method}{$z$-Test}
	Ein Test für den Erwartungswert bei bekannter Varianz. Seien $X_1,\ldots,X_n\sim \mathcal{N}(\mu,\sigma^2)$ unter $P_{\vartheta}$ mit bekanntem $\sigma^2$. Es ist $H_0 : \vartheta = \vartheta_0$ Möglich Variationen sind:
	\begin{itemize}
		\item $H_A: \vartheta > \vartheta_0$: $K = (c_>,\infty)$, $c_> = z_{(1-\alpha)}$
		\item $H_A: \vartheta < \vartheta_0$: $K = (-\infty,-c_<)$, $c_< = z_\alpha$.
		\item $H_A: \vartheta \neq \vartheta_0$: $K = (-\infty,-c_{\neq}) \cup (c_{\neq},\infty)$, $c_{\neq} = z_{1-\frac{\alpha}{2}}$.
	\end{itemize}
Mit $T=\frac{\overline{X_n}-\vartheta_0}{\sigma / \sqrt{n}} \sim \mathcal{N}(0,1)$ sowie $z_{x} = \Phi^{-1}(x)$.
\end{method}
Beachte dass $\overline{X_n} = \frac{1}{n} \sum_{i=1}^{n} X_i$

\begin{method}{$t$-Test}
	Ein Test für den Erwartungswert bei unbekannter Varianz. Seien $X_1,\ldots,X_n\sim \mathcal{N}(\mu,\sigma^2)$ unter $P_{\vec{\vartheta}}$ mit $\vec{\vartheta}=(\mu,\sigma^2)$. Wir haben $\Theta_0= \{\sigma\}\times(0,\infty)$. Die Teststatistik ist gegeben durch $T = \frac{\overline{X_n}- \mu_0}{S/\sqrt{n}} \sim t_{n-1}$ mit $S^2=\frac{1}{n-1}\sum_{i=1}^{n}\left(X_i-\overline{X_n}\right)^2$ für $\sigma^2$
	
	\begin{itemize}
		\item $H_A: \vartheta > \vartheta_0$: $K = (c_>,\infty)$, $c_> = t_{n-1,1-\alpha}$
		\item $H_A: \vartheta < \vartheta_0$: $K = (-\infty,c_<)$, $c_< = t_{n-1,\alpha} = -t_{n-1,1-\alpha}$.
		\item $H_A: \vartheta > \vartheta_0$: $K = (-\infty,-c_{\neq}) \cup (c_{\neq},\infty)$, $c_{\neq} = t_{n-1,1-\frac{\alpha}{2}}$.
	\end{itemize}
	wobei $t_{m,\gamma}$ das $\gamma$-Quantil der $t_m$ Verteilung ist.
\end{method}

\begin{method}{Gepaarte Zweistichproben-Test bei Normalverteilung}
	Wenn $X_1,\ldots,X_n$ i.i.d $\sim \mathcal{N}(\mu_X,\sigma^2)$ sowie $Y_1,\ldots,Y_n$ i.i.d $\sim \mathcal{N}(\mu_Y,\sigma^2)$ mit gleicher Anzahl Proben und gleicher Varianz, so definiert man $Z_i = X_i-Y_i$ die unter $P\vartheta$ i.i.d zu $\sim \mathcal{N}(\mu_X-\mu_Y,2\sigma^2)$ sind. Dann verwende entweder $z$-Test oder $t$-Test.
\end{method}


\begin{method}{Ungepaarter Zweistichproben-Test bei Normalverteilung}
	Seien unter $P\vartheta$ $X_1,\ldots,X_n$ i.i.d. $\sim \mathcal{N}(\mu_X,\sigma^2)$ sowie $Y_1,\ldots,Y_m$ i.i.d. $\sim \mathcal{N}(\mu_Y,\sigma^2)$ wobei $m\neq n$ sein kann.
	\begin{itemize}
		\item \textit{Ungepaarter Zweistichproben-$z$-Test wenn $\sigma^2$ bekannt ist:} Es ist $T = \frac{(\overline{X_n} -\overline{Y_m}) - (\mu_X - \mu_Y)}{\sigma \sqrt{\frac{1}{n} + \frac{1}{m}}} \sim \mathcal{N}(0,1)$
		\item \textit{Ungepaarter Zweistichproben-$t$-Test wenn $\sigma^2$ unbekannt ist:} Wir haben die beiden empirischen Varianzen
		$S^2_X = \frac{1}{n-1} \sum_{i=1}^{n} (X_i - \overline{X_n})^2$ und $S^2_Y = \frac{1}{n-1} \sum_{i=1}^{n} (Y_i - \overline{Y_n})^2$. Es gilt $S^2 = \frac{1}{m+n-2}\left(\left(n-1\right)S_X^2 + \left(m-1\right)S_Y^2\right)$ mit der Teststatistik $T = \frac{(\overline{X_n} -\overline{Y_m}) - (\mu_X - \mu_Y)}{S \sqrt{\frac{1}{n} + \frac{1}{m}}} \sim t_{n+m-2}$
	\end{itemize}
	
\end{method}


\section{Konfidenzbereiche}

\begin{method}{Konfidenzbereich}
Ein Konfidenzbereich ist eine Menge $C(X_1,\ldots,X_n) \subseteq \Omega$ zum Niveau $1-\alpha$ falls gilt

\begin{align*}
	P_{\vartheta} \left[\vartheta \in C(X_1,\ldots,X_n) \right] \geq 1 - \alpha \quad \text{für alle } \vartheta \in \Omega
\end{align*}
\end{method}


\begin{thm}{Bekannte Konfidenzbereiche}
\begin{itemize}
	\item 	$\mu$ unbekannt, $\sigma^2$ bekannt:
	\begin{itemize}
		\item $\mu \text{ zum Niveau } 1-\alpha$:
		\begin{align*}
			{\displaystyle \left[{{\overline{X_n}}-z_{(1-{\tfrac {\alpha }{2}})}{\frac {\sigma }{\sqrt {n}}}\ ;\ {\overline{X_n}}+z_{(1-{\tfrac {\alpha }{2}})}{\frac {\sigma }{\sqrt {n}}}}\right]}
		\end{align*}
	\end{itemize}
	\item 	$\mu, \sigma^2$ unbekannt:
	\begin{itemize}
\item $\mu \text{ zum Niveau } 1-\alpha$:
	\begin{align*}
  \left[ \overline{X_n} -t_{n-1,1-\frac{\alpha}{2}}\frac{S}{\sqrt{n}} , \overline{X_n} +t_{n-1,1-\frac{\alpha}{2}}\frac{S}{\sqrt{n}}   \right]
	\end{align*}
	\item $\sigma^2 \text{ zum Niveau } 1-\alpha$:
	\begin{align*}
\left[ \frac{(n-1)S^2}{\chi_{n-1,1-\frac{\alpha}{2}}^2} , \frac{(n-1)S^2}{\chi_{n-1,\frac{\alpha}{2}}^2}\right]
	\end{align*}
	\end{itemize}
\end{itemize}
\end{thm}

\textbf{Konstruktion von Konfidenzintervallen}

Angenommen wir haben 8 Gewichte gegeben die wir als Realisationen von $X_1,\ldots,X_8$ i.i.d. $\sim \mathcal{N}(\mu,\sigma^2)$. Als Schätzer verwenden wir $\mu = \overline{X_n} = \frac{1}{n} \sum_{i=1}^{n} X_i$ und als Stichprobenvarianz $S^2 = \frac{1}{n-1} \sum_{i=1}^{n} (X_i - \overline{X_n})^2$.

Wir machen den Ansatz $C(X_1,\ldots,X_n) \ = \left[\overline{X_n} - \cdots , \overline{X_n} + \cdots \right]$ und wollen erreichen dass $1-\alpha \leq 	P_{\vartheta} \left[\vartheta \in C(X_1,\ldots,X_n) \right] = 	P_{\vartheta} \left[\mu  \in \left[\overline{X_n} - \cdots , \overline{X_n} + \cdots\right] \right] = P_{\vartheta} \left[ |\overline{X_n} - \mu| \leq \cdots \right]$

Da $\frac{\overline{X_n} - \mu}{S/\sqrt{n}} \sim t_{n-1}$ brauchen wir $\frac{...}{S/\sqrt{n}} = t_{n-1,1-\frac{\alpha}{2}}$.

Dies gibt uns dann das Konfidentintervall für $\mu$ zum Niveau $1-\alpha$:

\begin{align*}
C(X_1,\ldots,X_n) &= \left[ \overline{X_n} -t_{n-1,1-\frac{\alpha}{2}}\frac{S}{\sqrt{n}} , \overline{X_n} +t_{n-1,1-\frac{\alpha}{2}}\frac{S}{\sqrt{n}}   \right]
\end{align*}

Um ein Konfidenzintervall für $\sigma^2$ zu konstruieren verwenden wir $\frac{1}{\sigma^2} (n-1)S^2 = \frac{1}{\sigma^2}\sum_{i=1}^{n} (X_i- \overline{X_n})^2 \sim \chi_{n-1}^2$

Wir rechnen wieder $1-\alpha = P_{\vartheta} \left[ \chi_{n-1,\frac{\alpha}{2}}^2 \leq \frac{1}{\sigma^2} (n-1)S^2 \leq \chi_{n-1,1-\frac{\alpha}{2}}^2 \right] = P_{\vartheta} \left[ \frac{(n-1)S^2}{\chi_{n-1,1-\frac{\alpha}{2}}^2} \leq \sigma^2 \leq \frac{(n-1)S^2}{\chi_{n-1,\frac{\alpha}{2}}^2}  \right]$

Dies impliziert 
\begin{align*}
	C(X_1,\ldots,X_n) = \left[ \frac{(n-1)S^2}{\chi_{n-1,1-\frac{\alpha}{2}}^2} , \frac{(n-1)S^2}{\chi_{n-1,\frac{\alpha}{2}}^2}\right]
\end{align*}
\section {Kombinatorik}

Wir definieren die 

\begin{itemize}
	\item \textbf{Auf wie viele Arten kann man $n$ Objekte (z.B. nebeneinander) anordnen?}
	
	Dies ist die Anzahl Permutationen von $n$ Elementen und ist $n!$.
	
	
	\item 	\textbf{Auf wie viele Arten kann man $k$ aus $n$ Objekten auswählen (mit $k\leq n$ ohne Zurücklegen)?}
	
	Dies ist die Anzahl Kombinationen ist $\binom{n}{k} = \frac{n!}{k!(n-k)!}$.
	
	
	\item \textbf{Wie viele Sequenzen der Länge $m$ kann ma mit den $n$ Symbolen bilden?}
	
	Dies ist die Anzahl der Variationen (mit Wiederholung) und ist $n^m$
\end{itemize}









%    _               _           _     
%   /_\  _ __   __ _| |_   _ ___(_)___ 
%  //_\\| '_ \ / _` | | | | / __| / __|
% /  _  \ | | | (_| | | |_| \__ \ \__ \
% \_/ \_/_| |_|\__,_|_|\__, |___/_|___/
%                       |___/           




% -------------------------- Ableitung --------------------------


\part{Analysis}

\section{Ableitung}

\begin{method}{Ableitung}
	Sei $D \subset \R$ , $f:D \to  \R$ und $x_0$ ein Häufungspunkt von $D$. $f$ ist in $x_0$ differenzierbar, falls der Grenzwert 
	$$ \lim\limits_{x \to x_0} \frac{f(x) -f(x_0)}{x-x_0}$$
	existiert. Ist dies der Fall, wird der Grenzwert mit $f'(x_0)$ bezeichnet.\\
	Alternativ nutzt man auch $x = x_0 + h$
	\begin{align*}
			f'(x_0) = \lim\limits_{h \to 0} \frac{f(x_0 + h) - f(x_0)}{h}
	\end{align*}
\end{method}



\subsection*{Konvexität}

\begin{method}{Konvex}
	$f : I \to \R$ ist konvex (auf I) falls für alle $x \leq y$ $x,y \in I$ und $\lambda \in [0,1]$
	\begin{align*}
	f(\lambda x + (1-\lambda)y) \leq \lambda f(x) + (1-\lambda)f(y)
	\end{align*}
	Zudem gilt für $x_0 < x< x_1$ in $I$:
	\begin{align*}
	 \frac{f(x) - f(x_0)}{x-x_0} \leq \frac{f(x_1) - f(x)}{x_1 - x}
	\end{align*}
	Man beweist dies indem man $x = (1-\lambda) x_0 + \lambda x_1$ wählt und somit $\lambda = \frac{x-x_0}{x_1 - x_0}$
\end{method}




\subsection*{Wichtige Taylorapproximationen um $x=0$}
\begin{itemize}
	\item $\boxed{\frac{1}{1-x}}$ Für alle $x \in (1,0)$ gilt:
	\begin{align*}
	{\frac{1}{1-x}} &= 1 + x + x^2 + x^3 + x^4 + \cdots \\
	&= \sum_{n=0}^{\infty} x^n
	\end{align*}	
	
	\item $\boxed{e^x}$ Für alle $x \in \R$ gilt:
	\begin{align*}
		e^x &= 1 + x + \frac{x^2}{2!} + \frac{x^3}{3!} + \frac{x^4}{4!}\\
		&= \sum_{n=0}^{\infty} \frac{x^n}{n!}
	\end{align*}
	
	\item $\boxed{\cos(x)}$ Für alle $x\in R$ gilt:
	\begin{align*}
	\cos(x) &= 1 - \frac{x^2}{2!} + \frac{x^4}{4!} - \frac{x^6}{6!} + \frac{x^8}{8!} - \cdots  \\
	&= \sum_{n=0}^{\infty} (-1)^n \frac{x^{2n}}{(2n)!}
	\end{align*}
	
	\item $\boxed{\sin(x)}$ Für alle $x\in R$ gilt:
	\begin{align*}
	\sin(x) &=  x - \frac{x^3}{3!} + \frac{x^5}{5!} - \frac{x^7}{7!} + \frac{x^9}{9!} - \cdots\\
	&= \sum_{n=0}^{\infty} (-1)^n \frac{x^{2n+1}}{(2n+1)!} = \sum_{n=1}^{\infty} (-1)^{(n-1)} \frac{x^{2n-1}}{(2n-1)!}
	\end{align*}

	\item $\boxed{\ln(1+x)}$ Für alle $x\in (-1,1]$ gilt:
	\begin{align*}
	\ln(x+1) &= x - \frac{x^2}{2} + \frac{x^3}{3} - \frac{x^4}{4} + \frac{x^5}{5}- \cdots \\
	&= \sum_{n=1}^{\infty} (-1)^{(n+1)} \frac{x^n}{n}
	\end{align*}

	\item $\boxed{\arctan(x)}$ Für alle $x\in [-1,1]$ gilt:
	\begin{align*}
	\arctan(x) &= x - \frac{x^3}{3} + \frac{x^5}{5} - \frac{x^7}{7} + \frac{x^9}{9} - \cdots \\
	&= \sum_{n=0}^{\infty} (-1)^n \frac{x^{2n+1}}{2n+1}
	\end{align*}

	\item $\boxed{(1 + x)^\alpha}$ Für alle $x\in \R$ gilt:
	\begin{align*}
	(1 + x)^\alpha &=  1 + \alpha x + \frac{\alpha(\alpha-1)}{2!} x^2 + \cdots \\
	 &= \sum_{k=0}^{\infty} \; {\alpha \choose k} \; x^k 
	\end{align*}
	
	\item $\boxed{\sinh(x)}$ Für alle $x\in \R$ gilt:
	\begin{align*}
	\sinh(x) &= x + \frac{x^3}{6} + \frac{x^5}{120} + \mathcal{O}(x^7)\\
	&= \sum_{k=0}^{\infty}\frac{x^{1+2k}}{(1+2k)!}
	\end{align*}
	
	\item $\boxed{\cosh(x)}$ Für alle $x\in \R$ gilt:
	\begin{align*}
	\cosh(x) &= 1 + \frac{x^2}{2} + \frac{x^4}{24} + \frac{x^6}{720} +  \mathcal{O}(x^7)\\
	&= \sum_{k=0}^{\infty}\frac{x^{2k}}{(2k)!}
	\end{align*}
	

\end{itemize}






\subsection*{Fundamentalsatz}

\begin{method}{Fundamentalsatz der Differentialrechnung}
Sei $f:[a,b] \to \R$ stetig. Dann gibt es eine Stammfunktion $F$ von $f$, die bis auf eine additive Konstante eindeutig bestimmt ist und es gilt:
$$\int_{a}^{b} f(x) dx = F(b) - F(a)$$
\textit{Beweis:} Existenz folgt aus Stammfunktionssatz. Seien $F_1, F_2$ Stammfkt., dann gilt $F_1' - F_2' = 0$. Somit ist $F_1 - F_2 = C$ mit $F(x) = C + \int_{a}^{x} f(t) dt$. Es folgt auch $F(a) = \int_{a}^{a} f(t) dt + C$ und somit $F(a) = C$. Es folgt daraus $F(b)-F(a) = \int_{a}^{b} f(t) dt$ \qed 
\end{method}


\subsection*{Ableitung des Integrals}
Mit der Kettenregel folgt aus dem Fundamentalsatz:
\begin{align*}
\frac{d}{dx} \left( \int_{u(x)}^{v(x)} f(t)  dt \right) &= f(v)\frac{dv}{dx} - f(u)\frac{du}{dx}
\end{align*}

\section*{Integrale Ausrechnen}

\begin{important}
Integrationskonstante $C$ nicht vergessen!
\end{important}

\subsection*{Direkte Integrale}
Diese sind vom Typ $\int f(g(x)) g'(x) dx = F(g(x))$.

\subsection*{Partielle Integration}
\begin{method}{Partielle Integration}
\begin{align*}
	\int f' \cdot g \ dx = f \cdot g - \int f \cdot g' \  dx
\end{align*}
\end{method}



\subsection*{Integrale rationaler Funktionen}
\begin{method}{Partielle Integration}
	$$\int \frac{p(x)}{q(x)} dx$$
	Wenn nun $\deg(p) \geq \deg(q)$ dann machen wir eine Polynomdivision $p:q$, sonst mache eine Parzialbruchzerlegung
\end{method}

\subsection*{Substitutionsregel}
\begin{method}{Substitutionsregel}
Ist $f$ stetig und $g$ erfüllt:
\begin{align*}
	y = g(x) \iff x = g^{-1}(y)
\end{align*}
Dann gilt:
\begin{align*}
\int_a ^b f(g(x))g'(x) dx &= \int_{g(a)}^{g(b)} f(y) dy
\end{align*}
Als Merksatz gilt $dy = g'(x) dx$ respektive $dx = \frac{1}{t} dt$
\end{method}

\subsubsection*{Integrale der Form $\int F(e^x, \sinh(x), \cosh(x)) dx$}
Substituiere mit $e^x = t$, ($dx = \frac{1}{t} dt$)\\
\textbf{Beispiel:}
\begin{align*}
	\int \frac{e^{2x}}{e^x + 1} dx &= \int \frac{t^2}{t + 1 } \frac{1}{t} dt = \int\frac{t +1 - 1}{t+1} dt\\
	\int \frac{1}{\cosh(x)} dx &= \int \frac{1}{\frac{1}{2} (e^x + e^{-x})} dx = \int \frac{2}{t + \frac{1}{t}} \frac{1}{t} dt = \frac{2}{t^2 + 1} dt
\end{align*}

\subsubsection*{Integrale der Form $\int F(\log(x)) dx$}
Substituiere mit $\log(x) = t$, ($dx = e^t dt$)\\
\textbf{Beispiel:}
\begin{align*}
	\int (\log(x))^2 dx &= \int t^2 e^t dt = t^2 e^t - \int 2t e^t dt  \\
	&= x(\log(x))^2 -2x\log(x) + 2x + C
\end{align*}

\subsubsection*{Integrale der Form $\int F(\sqrt[\alpha]{Ax + B}) dx$}
Substituiere mit $t = \sqrt[\alpha]{Ax + B}$\\
\textbf{Beispiel:}
\begin{align*}
	\int \frac{1}{\sqrt{x} \sqrt{1-x}} &= \int \frac{1}{t \sqrt{1-t^2}} 2t dt = \int \frac{2}{\sqrt{1-t^2}}
\end{align*}

\subsubsection*{Integrale die $\sin, \cos, \tan$ in geraden Potenzen enthalten}
Substituiere mit $\tan(x) = t$, ($dx = \frac{1}{1+t^2} dt$). Es gilt zudem:

\begin{align*}
\sin^2(x) &= \frac{t^2}{1+t^2} & \cos^2(x) &= \frac{1}{1+t^2}
\end{align*}

\subsubsection*{Integrale die $\sin, \cos, \tan$ in ungeraden Potenzen enthalten}
Substituiere mit $\tan(\frac{x}{2}) = t$, ($dx = \frac{2}{1+t^2} dt$). Es gilt zudem:

\begin{align*}
\sin(x) &= \frac{2t}{1+t^2} & \cos(x) &= \frac{1-t^2}{1+t^2}
\end{align*}



\subsubsection*{Integrale mit $\sqrt{Ax^2 + Bx + C}$ im Nenner}
Mithilfe quadratischer Ergänzung auf einen der folgenden Fälle zurückführen:
\begin{align*}
\int \frac{1}{\sqrt{1-x^2}} dx &= \arcsin(x) + C\\
\int \frac{1}{\sqrt{x^2-1}} dx &= \operatorname{arcosh}(x) + C\\
\int \frac{1}{\sqrt{1+x^2}} dx &= \operatorname{arcsinh}(x) + C
\end{align*}




\subsubsection*{Integrale mit $\sqrt{Ax^2 + Bx + C}$ im Zähler}
Mithilfe quadratischer Ergänzung auf einen der folgenden Fälle zurückführen, dann substituieren
\begin{align*}
\int {\sqrt{1-x^2}} dx \quad &\text{subsitution: } x = \sin(t) \Leftarrow dx = \cos(t) dt\\
\int {\sqrt{x^2-1}} dx \quad &\text{subsitution: } x = \cosh(t) \Leftarrow dx = \sinh(t) dt \\
\int {\sqrt{1+x^2}} dx \quad &\text{subsitution: } x = \sinh(t) \Leftarrow dx = \cosh(t) dt
\end{align*}








% -------------------------- Sonstiges --------------------------





\section*{Sonstiges}
\begin{method}{Binomialsatz}
	$\forall x,y \in \mathbb{C}$, $n \geq 1$ gilt:
	$$(x+y)^n = \sum_{k=0}^{n} \binom{n}{k}x^k y^{n-k}$$
\end{method}


\begin{method}{Binomialkoeffizienten}
	\begin{align*}
		\binom{n}{k} &= \frac{n!}{(n-k)!k!}
	\end{align*}
\end{method}

\begin{method}{ABC / Mitternachtsformel}
	\begin{align*}
	\text{Gegeben: } & ax^2 + bx + c = 0\\
	\text{Lösung: } & x_{1,2} = \frac{-b \pm \sqrt{b^2 -4ac}}{2a}
	\end{align*}
\end{method}


\begin{method}{Logarithmus Regeln}
\begin{align*}
	 \log _{b}(x\cdot y) &= \log _{b}(x)+\log _{b}(y)\\
	 \log_{b} (M^k) &= k \cdot \log_b (M)
\end{align*}
\end{method}


\begin{method}{Summenformeln}
	\begin{align*}
	\sum _{{k=1}}^{n}k &= {\frac  {n(n+1)}{2}}\\
	\sum_{k=1}^n (2k-1) &= n^2\\
	\sum _{{k=1}}^{n}k^{2} &= {\frac  {n(n+1)(2n+1)}{6}}
	\end{align*}
\end{method}


\begin{method}{Gerade \& Ungerade Funktion}
	Eine Funktion heisst:
	\begin{itemize}
		\item \textsc{Gerade} wenn $f(-x) = f(x)$
		\item \textsc{Ungerade} wenn $f(-x) = - f(x)$
	\end{itemize}
	Dabei sind $f(x) = 1$, $f(x) = |x|$, $f(x)=x^2$, $f(x) = \cos(x)$ alles gerade Funktionen.\\
	Im Gegenzug sind $f(x) = sgn(x)$, $f(x) = x$, $f(x) = \tan(x)$, $f(x) = \sin(x)$ ungerade Funktionen.
\end{method}


\begin{method}{Injektiv}
	\begin{align*}
	&\forall x_1,x_2 \in M : f(x_1) = f(x_2) \implies x_1 = x_2\\
	\text{or }  &x_1 \not = x_2 \implies f(x_1) \not = f(x_2)
	\end{align*}
	\textbf{Surjektiv}
	\begin{align*}
	\forall y \in N \exists x \in M : y = f(x)
	\end{align*}
\end{method}

\textbf{Umkehrsatz - Beispiel} Zeige dass $x + e^x$ bijektiv von $\R$ auf $\R$ abbildet. Es gilt $f'(x) = 1 + e^x > 0$, somit ist $f$ streng monoton wachsend in $\R$ und Umkehrbar. Weil $\lim_{x \to -\infty} f(x) = - \infty$ und $\lim_{x \to \infty} f(x) = \infty$ ist $f$ bijektiv von $\R$ nach $\R$



\begin{method}{Kreuzprodukt}
	\begin{align*}
	\vec{a}\times\vec{b}=	\begin{pmatrix}a_1 \\ a_2 \\ a_3\end{pmatrix}
	\times
	\begin{pmatrix}b_1 \\ b_2 \\ b_3 \end{pmatrix} &=	\begin{pmatrix}
	a_2b_3 - a_3b_2 \\
	a_3b_1 - a_1b_3 \\
	a_1b_2 - a_2b_1
	\end{pmatrix}
	\end{align*}
\end{method}

\newpage

\subsection*{Wichtige Integrale}

\begin{itemize}	
	
	\item ${\displaystyle \int \sin ^{2}{ax}\,dx={\frac {x}{2}}-{\frac {1}{4a}}\sin 2ax+C={\frac {x}{2}}-{\frac {1}{2a}}\sin ax\cos ax+C}$
	
	\item ${\displaystyle \int \sin ^{n}{ax}\,dx=-{\frac {\sin ^{n-1}ax\cos ax}{na}}+{\frac {n-1}{n}}\int \sin ^{n-2}ax\,dx\qquad {\mbox{(for }}n>0{\mbox{)}}}$
	
	\item ${\displaystyle {\begin{aligned}\int x^{n}\sin ax\,dx&=-{\frac {x^{n}}{a}}\cos ax+{\frac {n}{a}}\int x^{n-1}\cos ax\,dx\end{aligned}}}$
	
	\item ${\displaystyle \int \cos ^{2}{ax}\,dx={\frac {x}{2}}+{\frac {1}{4a}}\sin 2ax+C={\frac {x}{2}}+{\frac {1}{2a}}\sin ax\cos ax+C}$
	
	\item ${\displaystyle \int \cos ^{n}ax\,dx={\frac {\cos ^{n-1}ax\sin ax}{na}}+{\frac {n-1}{n}}\int \cos ^{n-2}ax\,dx\qquad {\mbox{(for }}n>0{\mbox{)}}}$
	
	\item ${\displaystyle {\begin{aligned}\int x^{n}\cos ax\,dx&={\frac {x^{n}\sin ax}{a}}-{\frac {n}{a}}\int x^{n-1}\sin ax\,dx\end{aligned}}}$
	
	
	
	
	
	\item ${\displaystyle \int (\sin ax)(\cos ax)\,dx={\frac {1}{2a}}\sin ^{2}ax+C}$
	
	\item ${\displaystyle \int (\sin ^{n}ax)(\cos ax)\,dx={\frac {1}{a(n+1)}}\sin ^{n+1}ax+C\qquad {\mbox{(for }}n\neq -1{\mbox{)}}}$
	
	\item ${\displaystyle \int (\sin ax)(\cos ^{n}ax)\,dx=-{\frac {1}{a(n+1)}}\cos ^{n+1}ax+C\qquad {\mbox{(for }}n\neq -1{\mbox{)}}}$
	
	\item $ {\displaystyle {\begin{aligned}\int (\sin ^{n}ax)(\cos ^{m}ax)\,dx&=-{\frac {(\sin ^{n-1}ax)(\cos ^{m+1}ax)}{a(n+m)}}\\&+{\frac {n-1}{n+m}}\int (\sin ^{n-2}ax)(\cos ^{m}ax)\,dx\qquad {\mbox{(for }}m,n>0{\mbox{)}}\end{aligned}}} $
	
	\item $\int \sin^2(x) \cos^2(x) dx = \frac{1}{4}\int sin^2(2x) dx = \frac{1}{4} \int \frac{1-\cos(4x)}{2}dx = \frac{x}{8} - \frac{1}{8} \frac{\sin(4x)}{4} + C$  
	
\end{itemize}



\subsubsection*{Typische Integrale}

\begin{itemize}
	\item $\int \frac{1}{x} \,dx = \ln |x|$
	\item $\int \frac{1}{x^2} \,dx = -\frac{1}{x}$
	\item $\int \frac{1}{x+a} \,dx = \ln |x+a|$
	\item $\int \ln(x) \,dx = x(\ln(x) - 1)$
	\item $\int \ln(ax + b) \,dx = \frac{(a x+b) \ln (a x+b)-a x}{a}$
	\item $\int \frac{1}{(x+a)^2} \,dx = - \frac{1}{x+a}$
	\item $\int \frac{1}{\sqrt{x}} \,dx = 2 \sqrt{x}$
	\item $\int \sqrt{1-x^2} dx = \frac{\arcsin(x) + x \sqrt{1-x^2}}{2} + C$
	\item $\int \frac{1}{ax+b} \,dx = \frac{1}{a} \ln |ax+b|$
	\item $\int \frac{1}{1 + x^2} \,dx = \frac{1}{2} \ln |1 + x^2|$
	\item $\int(ax + b)^n \,dx = \frac{(ax + b)^{n+1}}{(n + 1)a}, (n \neq -1)$
	\item $\int x(ax+b)^n \,dx = \frac{(ax + b)^{n+2}}{(n+2)a^2} -
	\frac{b(ax+b)^{n+1}}{(n+1)a^2}$
	\item $\int \frac{ax + b}{px + q} \,dx = \frac{ax}{p} + \frac{bp - aq}{p^2} \ln
	|pq+q|$
	\item $\int \frac{1}{a^2 + x^2} \,dx = \frac{1}{a} \arctan(\frac{x}{a})$
	\item $\int \frac{1}{a^2 - x^2} \,dx = \frac{1}{2a} \ln \left | \frac{a+x}{a-x}
	\right |$
	\item $\int \sqrt{x} \,dx = \frac{2}{3}\sqrt{x^3}$
	%mühsamer kerl der teilweise in prüfungen verwendet wird. Kann man über subsitution von x mit sin(u) lösen.
	\item $\int \sqrt{1-x^2} \,dx = \frac{1}{2}\left( x\sqrt{1-x^2}+\frac{1}{\sin(x)} \right)$
	\item $\int a^{xb + c} \,dx = \frac{a^{bx + c}}{b \log(a)}$
\end{itemize}

\subsubsection*{Trionometrische Funktionen}
\begin{itemize}
	\item $\int \sin(ax) \,dx = -\frac{1}{a}\cos(ax)$
	\item $\int \cos(ax) \,dx = \frac{1}{a}\sin(ax)$
	\item $\int \sin(ax)^2 \,dx = \frac{x}{2} - \frac{sin(2ax)}{4a}$
	\item $\int \frac{1}{\sin^2 x} \,dx = -\cot x$
	\item $\int x \sin(ax) \,dx = \frac{\sin(ax)}{a^2} - \frac{x \cos(ax)}{a}$
	\item $\int \cos^2(ax) \,dx = \frac{x}{2} + \frac{\sin(2ax)}{4a}$
	\item $\int \frac{1}{\cos^2(x)} \,dx = \tan x$
	\item $\int \cos(ax) \,dx = \frac{\cos(ax)}{a^2} + \frac{x \sin(ax)}{a}$
	\item $\int \sin(ax) \cos(ax) \,dx = -\frac{\cos^2(ax)}{2a}$
	\item $\int \tan(ax) \,dx = - \frac{1}{a} \ln | \cos(ax) |$
\end{itemize}

\subsubsection*{Exponentialfunktion}
\begin{itemize}
	\item $\int e^{ax} \,dx = \frac{1}{a} e^{ax}$ 
	\item $\int x e^{ax} \,dx = e^{ax} \cdot \left ( \frac{ax - 1}{a^2} \right )$
	\item $\int x \ln(x) \,dx = \frac{1}{2} x^2 (\ln(x) - \frac{1}{2})$
	\item $\int_{-\infty}^\infty e^{-\frac{1}{a}x^2} \,dx = \sqrt{a \pi}$
\end{itemize}




\subsection*{Vektoranalysis}

\begin{align*}
	{\displaystyle \Delta f=\operatorname {div} \left(\operatorname {grad} \,f\right),}\\
	{\displaystyle \Delta f=\nabla \cdot (\nabla f)=(\nabla \cdot \nabla )f=\nabla ^{2}f.}\\
	\nabla f = \begin{pmatrix}
	\frac{\partial }{\partial x}\\
	\frac{\partial }{\partial y}\\
	\frac{\partial }{\partial z}
	\end{pmatrix} \cdot f
\end{align*}


\begin{thm}{Integral der Normalverteilung}
	\begin{align*}
		\int_{-\infty}^{\infty} e^{-a(x+b)^2} \ dx = \sqrt{\frac{\pi}{a}}
	\end{align*}
\end{thm}




\vfill\null
\columnbreak

% to have more vertical space in the table.

{\renewcommand{\arraystretch}{1.5}
	\begin{table}[]
		\begin{tabular}{@{} p{.25\textwidth} p{.3\textwidth} p{.45\textwidth} @{}}
			\toprule
			Funktion & Ableitung & Bemerkung / Regel\\ \midrule
			$x$ & $1$ &   \\
			$x^2$& $2x$ &   \\
			$x^n$& $n\cdot x^{n-1}$ & $n \in \R$  \\
			$\frac{1}{x} = x^{-1}$ & $- \frac{1}{x^2}$ & \\
			$\sqrt{x} = x^{\frac{1}{2}}$ & $\frac{1}{2\sqrt{x}}$ & \\ 
			$\sqrt[n]{x} = x^{\frac{1}{n}}$ & $\frac{x^{\frac{1}{n} -1 }}{n}$ &  $\int x^{1/n} dx = \frac{n x^{1/n + 1}}{n+1} + C$\\ 
			$e^x$ & $e^x$ & \\
			$a^x$ & $\ln(a) \cdot a^x$& \\
			$x^x = e^{x\log(x)}$ & $x^x \cdot (\log(x) + 1)$ & Kettenregel $e^{x\log(x)}$\\
			$\ln(x)$ & $\frac{1}{x}$ & \\
			$x\ln(x) - x$ & $\ln(x)$ &  \\ \midrule
			$\sin(x)$ & $\cos(x)$ & \\
			$\cos(x)$ & $- \sin(x)$ & \\ 
			$\tan(x) = \frac{\sin(x)}{\cos(x)}$ & $\frac{1}{\cos^2(x)} = 1 + \tan^2(x)$ &\\
			$\cot(x) = \frac{\cos(x)}{\sin(x)}$ & - $\frac{1}{\sin^2(x)}$ & \\ 
			$\arcsin(x)$ & $\frac{1}{\sqrt{1 - x^2}}$ & $ \arcsin : [-1,1] \to [-\frac{\pi}{2},\frac{\pi}{2}]$\\
			$\arccos(x)$ & $ - \frac{1}{\sqrt{1-x^2}}$ & $\arccos : [-1,1] \to [0, \pi]$\\
			$\arctan(x)$ & $\frac{1}{1+x^2}$ & $\arctan:(-\infty, \infty) \to (- \frac{\pi}{2},\frac{\pi}{2})$\\
			$\operatorname{arccot}(x)$ & $ - \frac{1}{1+x^2} $ & $\operatorname{arccot} : (-\infty, \infty) \to (0,\pi)$\\
			\midrule
			$\cosh(x)$ & $\sinh(x)$ &\\
			$\sinh(x)$ & $\cosh(x)$ & \\
			$\tanh(x)$ & $\frac{1}{\cosh^2(x)}$ & \\
			$\operatorname{arsinh}(x)$ & $\frac{1}{\sqrt{1+x^2}}$ & $\forall x \in R$\\
			$\operatorname{arcosh}(x)$ & $\frac{1}{\sqrt{x^2 - 1}}$ & $\forall x \in (1, \infty)$\\		  $\operatorname{artanh}(x)$ & $\frac{1}{1-x^2}$ & $\forall x \in (-1,1)$\\
			\midrule
			$g(x) \cdot h(x)$ & $g(x) \cdot h'(x) + g'(x) \cdot h(x)$ & Produktregel\\
			$\left(g(x)\right)^n$ & $n \cdot \left( g(x) \right)^{n-1} \cdot g'(x)$ & Potenzregel\\
			$\frac{g(x)}{h(x)}$ & $\frac{ g'(x) \cdot h(x) - g(x)\cdot h'(x)}{\left(h(x)\right) ^2}$ & Quotientenregel\\
			$h(g(x))$ & $h'(g(x)) \cdot g'(x)$ & Kettenregel\\
			\bottomrule
		\end{tabular}
	\end{table}
}





\end{multicols}
\end{document}
